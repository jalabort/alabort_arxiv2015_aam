\subsection{Type of Composition}
\label{sec:composition}

Assuming, for the time being, that the true appearance parameters $\mathbf{c}^*$ are known, the problem defined by Equation \ref{eq:ssd} reduces to a non-rigid image alignment problem \cite{Baker2004, Munoz2014} between the particular instance of the object present in the image and its optimal appearance reconstruction by the appearance model:
\begin{equation}
    \begin{aligned}
        \mathbf{p}^* & = \underset{\mathbf{p}}{\mathrm{arg\,min\;}} 
        \frac{1}{2}\left\| \mathbf{i}[\mathbf{p}] - \mathbf{a} \right\|^2 
    \label{eq:ssd_shape}
    \end{aligned}
\end{equation}
where $\mathbf{a} = \bar{\mathbf{a}} + \mathbf{A} \mathbf{c}^*$ is obtained by directly evaluating Equation \ref{eq:app_model} given the true appearance parameters $\mathbf{c}^*$.

CGD algorithms iteratively solve the previous non-linear optimization problem with respect to the shape parameters $\mathbf{p}$ by:
\begin{enumerate}
    \item Introducing an incremental warp $\mathcal{W}(\mathbf{x}; \Delta\mathbf{p})$ according to the particular composition scheme being used.
    \label{it:step_1}
    
    \item Linearizing the previous incremental warp around the identity warp $\mathcal{W}(\mathbf{x}; \Delta\mathbf{p}) = \mathcal{W}(\mathbf{x}; \mathbf{0}) = \mathbf{x}$.
    \label{it:step_2}
    
    \item Solving for the parameters $\Delta\mathbf{p}$ of the incremental warp.
    \label{it:step_3}

    \item Updating the current warp estimate by using an appropriate compositional update rule.
    \label{it:step_4}

    \item Going back to Step \ref{it:step_1} until a particular convergence criteria is met.
    \label{it:step_5}
\end{enumerate}

%Note that the linearization of the incremental warp around $\Delta\mathbf{p} = \mathbf{0}$ and its composition with the current warp estimate $\mathcal{W}(\mathbf{x}; \mathbf{p} \circ \mathcal{W}(\mathbf{x}; \Delta\mathbf{p})$ i

%The incremental warp is a warp of the same family as the curent $\mathcal{W}(\mathbf{x}, \mathbf{p})$ used to warp the input image. 

Existent CGD algorithm for fitting AAMs have introduced the incremental warp either on the image or the model sides in what are known as \emph{forward} and \emph{inverse} compositional frameworks \cite{Matthews2004, Gross2005, Papandreou2008, Amberg2009, Martins2010, Tzimiropoulos2013} respectively. Inspired by related works in field of image alignment \cite{Malis2004, Megret2008, Autheserre2009, Megret2010}, we notice that novel CGD algorithms can be derived by introducing incremental warps on both image and model sides simultaneously. Depending on the exact relationship between these incremental warps we define two novel types of composition: \emph{asymmetric} and \emph{bidirectional}.

The following subsections explain how to introduce the incremental warp into the cost function and how to update the current warp estimate for the four types of composition considered in this paper: 
\begin{inparaenum}[\itshape i\upshape)]
    \item forward; 
    \item inverse;
    \item asymmetric; and
    \item bidirectional.
\end{inparaenum} For convenience, in these subsections we will use the simplified cost function defined by Equation \ref{eq:ssd_shape}. Furthermore, to maintain consistency with the vector notation used through out the paper, we will abuse the notation and write the operations of warp composition\footnote{\label{foot:warp}Further details regarding composition, $\mathbf{p} \circ \Delta \mathbf{p}$, and inversion, $\Delta \mathbf{p}^{-1}$, of typical AAMs' motion models such as PWA and TPS warps can be found in \cite{Matthews2004, Papandreou2008}.} $\mathcal{W}(\mathbf{x}; \mathbf{p}) \circ \mathcal{W}(\mathbf{x}; \Delta\mathbf{p})$ and inversion\footnoteref{foot:warp} $\mathcal{W}(\mathbf{x}; \mathbf{p})^{-1}$ as simply $\mathbf{p} \circ \Delta \mathbf{p}$ and $\mathbf{p}^{-1}$ respectively.

\subsubsection{Forward}
\label{sec:forward}

In the forward compositional framework the incremental warp $\Delta \mathbf{p}$ is introduced on the image side at each iteration by composing it with the current warp estimate $\mathbf{p}_{k-1}^*$:
\begin{equation}
    \begin{aligned}
        \Delta \mathbf{p}^* & = \underset{\Delta \mathbf{p}} {\mathrm{arg\, min\;}} \frac{1}{2}|| \mathbf{i}[\mathbf{p} \circ \Delta \mathbf{p}] - \mathbf{a} ||^2
    \label{eq:ssd_fc}
    \end{aligned}
\end{equation}

Once the optimal values for the parameters of the incremental warp are obtained, the current warp estimate is updated according to the following compositional update rule:
\begin{equation}
 	\begin{aligned}
    	\mathbf{p} \leftarrow \mathbf{p} \circ \Delta \mathbf{p}
    \label{eq:fc_update}
    \end{aligned}
\end{equation}

\subsubsection{Inverse}
\label{sec:inverse}

On the other hand, the inverse compositional framework inverts the roles of the image and the model by introducing the incremental warp on the model side:
\begin{equation}
    \begin{aligned}
        \Delta \mathbf{p}^* & = \underset{\Delta \mathbf{p}} {\mathrm{arg\, min\;}} \frac{1}{2}|| \mathbf{i}[\mathbf{p}] - \mathbf{a} [\Delta \mathbf{p}] ||^2
    \label{eq:ssd_ic}
    \end{aligned}
\end{equation}
Note that, in this case, the model is the the one we seek to deform using the incremental warp.

Because the incremental warp is introduced on the model side, the solution $\Delta \mathbf{p}$ needs to be inverted before it is composed with the current warp estimate:
\begin{equation}
 	\begin{aligned}
    	\mathbf{p} \leftarrow \mathbf{p} \circ \Delta {\mathbf{p}}^{-1} 
    \label{eq:ic_update}
    \end{aligned}
\end{equation}

\subsubsection{Asymmetric}
\label{sec:asymmetric}

Asymmetric composition introduces two related incremental warps onto the cost function; one on the image side (forward) and the other on the model side (inverse): 
\begin{equation}
    \begin{aligned}
        \Delta \mathbf{p}^* & = \underset{\Delta \mathbf{p}} {\mathrm{arg\, min\;}} \frac{1}{2}|| \mathbf{i}[\mathbf{p} \circ \alpha \Delta \mathbf{p}] - \mathbf{a} [\beta \Delta \mathbf{p}^{-1}] ||^2
    \label{eq:ssd_ac}
    \end{aligned}
\end{equation}
Note that the previous two incremental warps are defined to be each others inverse. Consequently, using the first order approximation to warp inversion for typical AAMs warps $\Delta\mathbf{p}^{-1} = -\Delta\mathbf{p}$ defined in \cite{Matthews2004}, we can rewrite the previous asymmetric cost function as:
\begin{equation}
    \begin{aligned}
        \Delta \mathbf{p}^* & = \underset{\Delta \mathbf{p}} {\mathrm{arg\, min\;}} \frac{1}{2}|| \mathbf{i}[\mathbf{p} \circ \alpha \Delta \mathbf{p}] - \mathbf{a} [-\beta \Delta \mathbf{p} ||^2
    \label{eq:ssd_ac2}
    \end{aligned}
\end{equation}
Although this cost function will need to be linearized around both incremental warps, the parameters $\Delta \mathbf{p}$ controlling both incremental warps are the same. Also, note that the parameters $\alpha \in [0, 1]$ and $\beta=(1-\alpha)$ control the relative contribution of both incremental warps in the computation of the optimal value for $\Delta \mathbf{p}$. 

In this case, the update rule for the current warp estimate is obtained by combining the previous forward and inverse compositional update rules into a single compositional update rule:
\begin{equation}
 	\begin{aligned}
    	\mathbf{p} & \leftarrow \mathbf{p} \circ \alpha \Delta \mathbf{p} \circ \beta \Delta \mathbf{p}
    \label{eq:ac_update}
    \end{aligned}
\end{equation}

Note that, the special case in which $\alpha = \beta = 0.5$ is also referred to as \emph{symmetric} composition \cite{Megret2008, Autheserre2009, Megret2010} and that the previous forward and inverse compositions can also be obtained from asymmetric composition by setting $\alpha = 1$ , $\beta = 0$ and $\alpha = 0$ , $\beta = 1$ respectively.

\subsubsection{Bidirectional}
\label{sec:bidirectional}

Similar to the previous asymmetric composition, bidirectional composition also introduces incremental warps on both image and model sides. However, in this case, the two incremental warps are assumed to be independent from each other:
\begin{equation}
    \begin{aligned}
        \Delta \mathbf{p}^*, \Delta \mathbf{q}^*  & = \underset{\Delta \mathbf{p}, \Delta \mathbf{q}} {\mathrm{arg\, min\;}} \frac{1}{2}|| \mathbf{i}[\mathbf{p}\circ \Delta \mathbf{p}] - \mathbf{a} [\Delta \mathbf{q}] ||^2
    \label{eq:ssd_bc}
    \end{aligned}
\end{equation}
Consequently, in Step \ref{it:step_4}, the cost function needs to be linearized around both incremental warps and solved with respect to the parameters controlling both warps, $\Delta \mathbf{p}$ and $\Delta \mathbf{q}$. 

Once the optimal value for both sets of parameters is recovered, the current estimate of the warp is updated using:
\begin{equation}
 	\begin{aligned}
    	\mathbf{p} \leftarrow \mathbf{p} \circ \Delta \mathbf{p} \circ \Delta {\mathbf{q}}^{-1} 
    \label{eq:bc_update}
    \end{aligned}
\end{equation}
