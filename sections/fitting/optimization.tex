\subsection{Optimization Method}
\label{sec:optimization}
Steps \ref{it:step_2} i.e. linearizing the cost function around the incremental warp; and \ref{it:step_3} i.e solving for the parameters of the incremental warp; depend on the particular optimization method used by CGD algorithm.

In this paper we distinguish between three main optimization methods\footnote{Notice that Amberg et al. proposed the use the \emph{Steepest Descent} method \cite{Boyd2004} in \cite{Amberg2009}. However, their approach requires a special formulation of the motion model and it performs poorly using the standard independent AAM formulation \cite{Matthews2004} used in this paper.}:
\begin{inparaenum}[\itshape i\upshape)]
    \item \emph{Gauss-Newton} \cite{Boyd2004, Matthews2004, Gross2005, Martins2010};
    \item \emph{Newton} \cite{Boyd2004, Kossaifi2014}; and
    \item \emph{Wiberg} \cite{Okatani2006, Strelow2012, Papandreou2008, Tzimiropoulos2013}.
\end{inparaenum}
The previous methods can be used to iteratively solve the non-linear optimization problems defined by Equations \ref{eq:prob_rssd} and \ref{eq:prob_po}. The main differences between them are: 
\begin{enumerate}
    \item The cost function term being linearized. The Gauss-Newton and Wiberg methods linearize the residual term $\mathbf{r}$ while the Newton method linearizes the entire data term $\mathcal{D}$.
    
    \item The way in which the algorithms solve for the different sets of parameters $\Delta \mathbf{c}$, $\Delta \mathbf{p}$ and $\Delta \mathbf{q}$. The Gauss-Newton and Newton methods can either solve for them \emph{simultaneously} or in an \emph{alternate} fashion while the Wiberg method defines its own procedure to solve for different sets of parameters\footnote{Note that the Wiberg method reduces to the Gauss-Newton method when only a single set of parameters needs to be inferred.}.
\end{enumerate}

The following section detail how the previous optimization methods are used:

In order to simplify the comprehension of this section all derivation will be given using the SSD data term defined by Equation \ref{eq:ssd} and the asymmetric and bidirectional compositions presented in Sections \ref{sec:asymmetric} and \ref{sec:bidirectional}. These represent the most general\footnote{Notice that the derivations for the forward, inverse and symmetric compositions can be obtained from the asymmetric one by setting $\alpha = 1$, $\alpha = 0$ and $\alpha = 0.5$ respectively. Also, the derivations using the project-out data term, \ref{eq:po}, can be easily obtained from the SSD ones.} cases because they require solving for all sets of parameters, $\Delta \mathbf{c}$, $\Delta \mathbf{p}$ and $\Delta \mathbf{q}$.

These represents the most general cases because it requires solving for all parameters, $\Delta \mathbf{c}$, $\Delta \mathbf{p}$ and $\Delta \mathbf{q}$.

\subsubsection{Gauss-Newton}
\label{sec:gauss_newton}

When \emph{asymmetric} composition is used the optimization problem is defined as:
\begin{equation}
    \begin{aligned}
        \Delta \mathbf{c}^*, \Delta \mathbf{p}^* & = \underset{\Delta \mathbf{c}, \Delta \mathbf{p}}{\mathrm{arg\,min\;}} \mathbf{r}_a^T\mathbf{r}_a
    \label{eq:asymmetric_ssd}
    \end{aligned}
\end{equation}
where the asymmetric residual $\mathbf{r}_a$ is defined as: 
\begin{equation}
    \begin{aligned}
		\mathbf{r}_a & = \mathbf{i}[\mathbf{p}_{k-1}^* \circ \alpha \Delta \mathbf{p}] - (\mathbf{a} + \mathbf{A}(\mathbf{c}^*_{k-1} + \Delta\mathbf{c})) [\beta \Delta \mathbf{p}]
    \label{eq:asymmetric_residual}
    \end{aligned}
\end{equation}
The Gauss-Newton algorithm solves the previous optimization problem by first performing a first-order Taylor expansion of the residual around the incremental warps:
\begin{equation}
    \begin{aligned}
		\mathbf{r}_a(\Delta \boldsymbol{\ell}) & \approx \hat{\mathbf{r}}_a(\Delta \boldsymbol{\ell})
		\\
		& \approx \mathbf{r}_a + \frac{\partial \mathbf{r}_a}{\partial \Delta \boldsymbol{\ell}} \Delta \boldsymbol{\ell}
    \label{eq:asymmetric_residual_taylor}
    \end{aligned}
\end{equation}
and then solving the following approximation of the original optimization problem:
\begin{equation}
    \begin{aligned}
        \Delta \boldsymbol{\ell}^* & = \underset{\Delta \boldsymbol{\ell}}{\mathrm{arg\,min\;}} \hat{\mathbf{r}}_a^T\hat{\mathbf{r}}_a
    \label{eq:asymmetric_ssd_taylor}
    \end{aligned}
\end{equation}
where we have defined $\Delta \boldsymbol{\ell} = (\Delta \mathbf{c}^T, \Delta \mathbf{p}^T)^T$ and the partial derivative of the residual with respect to the parameters is defined as: 
\begin{equation}
    \begin{aligned}
		\frac{\partial \mathbf{r}_a}{\partial \Delta \boldsymbol{\ell}}& = \left( \frac{\partial \mathbf{r}_a}{\partial \Delta \mathbf{c}}, \frac{\partial \mathbf{r}_a}{\partial \Delta \mathbf{p}} \right)
		\\
		& = \left( \mathbf{A}, \left( \alpha \nabla \mathbf{i}[\mathbf{p}]  + \beta \nabla (\mathbf{a} + \mathbf{A}\mathbf{c}_{k-1}^*) \right) \frac{\partial \mathcal{W}}{\partial \Delta \mathbf{p}} \right)
		\\
		& = \left( \mathbf{A}, \nabla \mathbf{t} \frac{\partial \mathcal{W}}{\partial \Delta \mathbf{p}} \right)
		\\
		& = \left( \mathbf{A}, \mathbf{J}_\mathbf{t} \right)
    \label{eq:asymmetric_jacobian}
    \end{aligned}
\end{equation}
where, in order to unclutter the notation, we have defined $\nabla\mathbf{t} = \left( \alpha \nabla \mathbf{i}[\mathbf{p}]  + \beta \nabla (\mathbf{a} + \mathbf{A}\mathbf{c}_{k-1}^*) \right)$.

When \emph{bidirectional} composition is used the optimization problem is defined as:
\begin{equation}
    \begin{aligned}
        \Delta \mathbf{c}^*, \Delta \mathbf{p}^*, \Delta \mathbf{q}^*& = \underset{\Delta \mathbf{c}, \Delta \mathbf{p}, \Delta \mathbf{q}}{\mathrm{arg\,min\;}} \mathbf{r}_b^T\mathbf{r}_b
    \label{eq:bidirectional_ssd}
    \end{aligned}
\end{equation}
where the bidirectional residual $\mathbf{r}_b$ reduces to: 
\begin{equation}
    \begin{aligned}
		\mathbf{r}_b & = \mathbf{i}[\mathbf{p}_{k-1}^* \circ \Delta \mathbf{p}] - (\mathbf{a} + \mathbf{A}(\mathbf{c}^*_{k-1} + \Delta\mathbf{c})) [\Delta \mathbf{q}]
    \label{eq:bidirectional_residual}
    \end{aligned}
\end{equation}
The Gauss-Newton algorithm proceeds in exactly the same manner as before, performing a first order Taylor expansion:
\begin{equation}
    \begin{aligned}
		\mathbf{r}_b(\Delta \boldsymbol{\ell}) & \approx \hat{\mathbf{r}}_b(\Delta \boldsymbol{\ell})
		\\
		& \approx \mathbf{r}_b + \frac{\partial \mathbf{r}_b}{\partial \Delta \boldsymbol{\ell}} \Delta \boldsymbol{\ell}
    \label{eq:idirectional_residual_taylor}
    \end{aligned}
\end{equation}
and solving the approximation to the original problem:
\begin{equation}
    \begin{aligned}
        \Delta \boldsymbol{\ell}^* & = \underset{\Delta \boldsymbol{\ell}}{\mathrm{arg\,min\;}} \hat{\mathbf{r}}_b^T\hat{\mathbf{r}}_b
    \label{eq:bidirectional_ssd_taylor}
    \end{aligned}
\end{equation}
where, in this case, $\Delta \boldsymbol{\ell} = (\Delta \mathbf{c}^T, \Delta \mathbf{p}^T, \Delta \mathbf{q}^T)^T$ and the partial derivative of the residual is defined as: 
\begin{equation}
    \begin{aligned}
		\frac{\partial \mathbf{r}_b}{\partial \Delta \boldsymbol{\ell}}& = \left( \frac{\partial \mathbf{r}_b}{\partial \Delta \mathbf{c}}, \frac{\partial \mathbf{r}_b}{\partial \Delta \mathbf{p}}, \frac{\partial \mathbf{r}_b}{\partial \Delta \mathbf{q}} \right)
		\\
		& = \left( \mathbf{A}, \nabla \mathbf{i}[\mathbf{p}] \frac{\partial \mathcal{W}}{\partial \Delta \mathbf{p}}, \nabla (\mathbf{a} + \mathbf{A}\mathbf{c}_{k-1}^*) \frac{\partial \mathcal{W}}{\partial \Delta \mathbf{q}} \right)
		\\
		& = \left( \mathbf{A}, \mathbf{J}_{\mathbf{i}}, \mathbf{J}_{\mathbf{a}} \right)
    \label{eq:bidirectional_jacobian}
    \end{aligned}
\end{equation}

% \begin{equation}
%     \begin{aligned}
% 		\frac{\partial \mathbf{r}}{\partial \Delta \mathbf{c}} & = \mathbf{A}, \nabla \mathbf{i}[\mathbf{p}] \frac{\partial \mathcal{W}}{\partial \mathbf{p}}, 
%     \label{eq:ssd_bc}
%     \end{aligned}
% \end{equation}

% \begin{equation}
%     \begin{aligned}
% 		\frac{\partial \mathbf{r}}{\partial \Delta \mathbf{p}} & = \mathbf{J}_\mathbf{i}
% 		\\
% 		& = \nabla \mathbf{i}[\mathbf{p}] \frac{\partial \mathcal{W}}{\partial \mathbf{p}}
%     \label{eq:ssd_bc}
%     \end{aligned}
% \end{equation}

% \begin{equation}
%     \begin{aligned}
% 		\frac{\partial \mathbf{r}}{\partial \Delta \mathbf{q}} & = \mathbf{J}_\mathbf{a}
% 		\\
% 		\\
% 		& = \nabla (\mathbf{a} + \mathbf{A}\mathbf{c}_{k-1}^*) \frac{\partial \mathcal{W}}{\partial \mathbf{p}}
%     \label{eq:ssd_bc}
%     \end{aligned}
% \end{equation}

% \begin{equation}
%     \begin{aligned}
% 		\frac{\partial \mathbf{r}}{\partial \Delta \boldsymbol{\ell}}\Delta \boldsymbol{\ell} & = (\mathbf{A}, \mathbf{J}_\mathbf{i}, \mathbf{J}_\mathbf{a}) 
% 		\begin{pmatrix}
% 			\Delta \mathbf{c}
% 			\\
% 			\Delta \mathbf{p}
% 			\\
% 			\Delta \mathbf{q}
% 		\end{pmatrix}
%     \label{eq:ssd_bc}
%     \end{aligned}
% \end{equation}

\subsubsection*{Simultaneous}
\label{sec:gauss_newton_simultaneous}

The optimization problem defined by Equations \ref{eq:asymmetric_ssd_taylor}  and \ref{eq:bidirectional_ssd_taylor} can be solve with respect to all parameters simultaneously by simply equating their derivative to $0$:
\begin{equation}
    \begin{aligned}
		0 & = \frac{\partial\hat{\mathcal{D}}}{\partial \Delta \boldsymbol{\ell}}
		\\
		& = \frac{\partial\frac{1}{2}\hat{\mathbf{r}}^T \hat{\mathbf{r}}}{\partial \Delta \boldsymbol{\ell}}
		\\
		& = \frac{\partial\frac{1}{2}(\mathbf{r} + \frac{\partial \mathbf{r}}{\partial \Delta \boldsymbol{\ell}} \Delta \boldsymbol{\ell})^T(\mathbf{r} + \frac{\partial \mathbf{r}}{\partial \Delta \boldsymbol{\ell}} \Delta \boldsymbol{\ell})}{\partial \Delta \boldsymbol{\ell}}
		\\
		& = \left( \mathbf{r} + \frac{\partial \mathbf{r}}{\partial \Delta \boldsymbol{\ell}} \Delta \boldsymbol{\ell} \right) \frac{\partial \mathbf{r}}{\partial \Delta \boldsymbol{\ell}}^T
    \label{eq:ssd_bc}
    \end{aligned}
\end{equation}
and the solution given by:
\begin{equation}
    \begin{aligned}
		\Delta \boldsymbol{\ell} & =  \left( \frac{\partial \mathbf{r}}{\partial \Delta \boldsymbol{\ell}}^T \frac{\partial \mathbf{r}}{\partial \Delta \boldsymbol{\ell}} \right)^{-1} \frac{\partial \mathbf{r}}{\partial \Delta \boldsymbol{\ell}}^T \mathbf{r}
    \label{eq:sim_solution}
    \end{aligned}
\end{equation}

Note that, solving the previous equation by directly inverting $\left( \frac{\partial \mathbf{r}}{\partial \Delta \boldsymbol{\ell}}^T \frac{\partial \mathbf{r}}{\partial \Delta \boldsymbol{\ell}} \right)$ has complexity $O((m + 2n)^3)$. However, as noted in \cite{Papandreou2008}\footnote{Notice that the authors of \cite{Papandreou2008} applied the Schur complement using inverse composition while we apply it here using the more general \emph{asymmetric} (which includes \emph{forward}, \emph{inverse} and \emph{symmetric}) and \emph{bidirectional} compositions.}, one can take advantage of the inherent structure of the problem and derive an algorithm with smaller complexity by making use of the Schur complement. 

For \emph{asymmetric} composition we have: 
\begin{equation}
    \begin{aligned}
    	\left( \frac{\partial \mathbf{r}_a}{\partial \Delta \boldsymbol{\ell}}^T \frac{\partial \mathbf{r}_a}{\partial \Delta \boldsymbol{\ell}} \right) \Delta \boldsymbol{\ell} & = \frac{\partial \mathbf{r}_a}{\partial \Delta \boldsymbol{\ell}}^T \mathbf{r}
    	\\
        \begin{pmatrix}
            \underbrace{\mathbf{A}^T \mathbf{A}}_{\mathbf{I}} & \mathbf{A}^T \mathbf{J}_{\mathbf{t}}
            \\ 
            \mathbf{J}_a^T \mathbf{A} & \mathbf{J}_{\mathbf{t}}^T \mathbf{J}_{\mathbf{t}}
        \end{pmatrix}
        \begin{pmatrix}
            \Delta\mathbf{c} 
            \\ 
            \Delta\mathbf{p}
        \end{pmatrix}
        & = 
        \begin{pmatrix}
            \mathbf{A}^T
            \\ 
            \mathbf{J}_{\mathbf{t}}^T
        \end{pmatrix} \mathbf{r}_a
    \label{eq:asymmetric_structure}
    \end{aligned}
\end{equation}
Applying the Schur complement the solution for $\Delta\mathbf{p}$ is given by:
\begin{equation}
    \begin{aligned}
        (\mathbf{J}_{\mathbf{t}}^T\mathbf{J}_{\mathbf{t}} - \mathbf{J}_{\mathbf{t}}^T\mathbf{A}\mathbf{A}^T\mathbf{J}_{\mathbf{t}}^T) \Delta \mathbf{p} & = \mathbf{J}_{\mathbf{t}}^T\mathbf{r} - \mathbf{J}_{\mathbf{t}}^T\mathbf{A} \mathbf{A}^T \mathbf{r}_a
        \\
        \mathbf{J}_{\mathbf{t}}^T(\mathbf{I} - \mathbf{A} \mathbf{A}^T)\mathbf{J}_{\mathbf{t}} \Delta \mathbf{p} & = \mathbf{J}_{\mathbf{t}}^T(\mathbf{I} - \mathbf{A} \mathbf{A}^T)\mathbf{r}_a
        \\
        \mathbf{J}_{\mathbf{t}}^T\bar{\mathbf{A}}\mathbf{J}_{\mathbf{t}} \Delta \mathbf{p} & = \mathbf{J}_{\mathbf{t}}^T\bar{\mathbf{A}}\mathbf{r}_a
        \\
        \Delta \mathbf{p} & = \left( \mathbf{J}_{\mathbf{t}}^T\bar{\mathbf{A}}\mathbf{J}_{\mathbf{t}} \right)^{-1} \mathbf{J}_{\mathbf{t}}^T\bar{\mathbf{A}}\mathbf{r}_{\mathbf{t}}
    \label{eq:asymmetric_schur_solution1}
    \end{aligned}
\end{equation}
and plugging the solution for $\Delta\mathbf{p}$ into equation \ref{eq:asymmetric_structure} the optimal value for $\Delta\mathbf{c}$ is obtained by:
\begin{equation}
    \begin{aligned}
        \Delta \mathbf{c} + \mathbf{A}^T\mathbf{J}_{\mathbf{t}}\Delta\mathbf{p} & = \mathbf{A}^T
        \\
        \Delta \mathbf{c} & = \mathbf{A}^T \left( \mathbf{r}_a - \mathbf{J}_{\mathbf{t}} \Delta\mathbf{p} \right)
    \label{eq:asymmetric_schur_solution2}
    \end{aligned}
\end{equation}

Using \emph{bidirectional} composition we can apply the Schur complement either ones or twice in order to take advantage of the $3\times3$ block structure of the matrix $\left( \frac{\partial \mathbf{r}_b}{\partial \Delta \boldsymbol{\ell}}^T \frac{\partial \mathbf{r}_b}{\partial \Delta \boldsymbol{\ell}} \right)$:
\begin{equation}
    \begin{aligned}
    	\left( \frac{\partial \mathbf{r}_b}{\partial \Delta \boldsymbol{\ell}}^T \frac{\partial \mathbf{r}_b}{\partial \Delta \boldsymbol{\ell}} \right) \Delta \boldsymbol{\ell} & = \frac{\partial \mathbf{r}_b}{\partial \Delta \boldsymbol{\ell}}^T \mathbf{r}_b
    	\\
        \left(\begin{array}{c|cc}
            \underbrace{\mathbf{A}^T \mathbf{A}}_{\mathbf{I}} & \mathbf{A}^T \mathbf{J}_{\mathbf{i}} & \mathbf{A}^T \mathbf{J}_{\mathbf{a}} 
            \\ \hline
            \mathbf{J}_{\mathbf{i}}^T \mathbf{A} & \mathbf{J}_{\mathbf{i}}^T \mathbf{J}_{\mathbf{i}} & \mathbf{J}_{\mathbf{i}}^T \mathbf{J}_{\mathbf{a}}
            \\ 
            \mathbf{J}_{\mathbf{a}}^T \mathbf{A} & \mathbf{J}_{\mathbf{a}}^T \mathbf{J}_{\mathbf{i}} & \mathbf{J}_{\mathbf{a}}^T \mathbf{J}_{\mathbf{a}}  
        \end{array} \right)
        \begin{pmatrix}
            \Delta\mathbf{c} 
            \\ \hline
            \Delta\mathbf{p}
            \\ 
            \Delta\mathbf{q}
        \end{pmatrix}
        & = 
        \begin{pmatrix}
            \mathbf{A}^T
            \\ \hline
            \mathbf{J}_{\mathbf{i}}^T
            \\ 
            \mathbf{J}_{\mathbf{a}}^T
        \end{pmatrix} \mathbf{r}_b
    \label{eq:bidirectional_structure}
    \end{aligned}
\end{equation}
Applying the Schur complement once, the combined solution for $(\Delta\mathbf{p}^T, \Delta\mathbf{q}^T)^T$ is given by:
\begin{equation}
    \begin{aligned}
        \begin{pmatrix}
            \mathbf{J}_{\mathbf{i}}^T \bar{\mathbf{A}}\mathbf{J}_{\mathbf{i}} & \mathbf{J}_{\mathbf{i}}^T \bar{\mathbf{A}}\mathbf{J}_{\mathbf{a}}
            \\ 
            \mathbf{J}_{\mathbf{a}}^T \bar{\mathbf{A}}\mathbf{J}_{\mathbf{i}} & \mathbf{J}_{\mathbf{a}}^T \bar{\mathbf{A}}\mathbf{J}_{\mathbf{a}}
            \\
        \end{pmatrix}
        \begin{pmatrix}
            \Delta\mathbf{p}
            \\
            \Delta\mathbf{q}
        \end{pmatrix} & = 
        \begin{pmatrix}
            \bar{\mathbf{A}}\mathbf{J}_{\mathbf{i}}^T
            \\
            \bar{\mathbf{A}}\mathbf{J}_{\mathbf{a}}^T
        \end{pmatrix} \mathbf{r}_b
       \\
       \begin{pmatrix}
            \Delta\mathbf{p}
            \\
            \Delta\mathbf{q}
        \end{pmatrix} & = 
        \begin{pmatrix}
            \mathbf{J}_{\mathbf{i}}^T \bar{\mathbf{A}}\mathbf{J}_{\mathbf{i}} & \mathbf{J}_{\mathbf{i}}^T \bar{\mathbf{A}}\mathbf{J}_{\mathbf{a}}
            \\ 
            \mathbf{J}_{\mathbf{a}}^T \bar{\mathbf{A}}\mathbf{J}_{\mathbf{i}} & \mathbf{J}_{\mathbf{a}}^T \bar{\mathbf{A}}\mathbf{J}_{\mathbf{a}}
            \\
        \end{pmatrix}^{-1}
        \\
        & \quad \,
        \begin{pmatrix}
            \bar{\mathbf{A}}\mathbf{J}_{\mathbf{i}}^T
            \\
            \bar{\mathbf{A}}\mathbf{J}_{\mathbf{a}}^T
        \end{pmatrix} \mathbf{r}_b
    \label{eq:bidirectional_schur_solution1}
    \end{aligned}
\end{equation}
Note that the complexity of inverting the previous Gauss-Newton approximation to the Hessian matrix has reduced to $O((2n)^3)$\footnote{This is an important reduction in complexity because usually $m >> n$ in CGD algorithms.} and, similar as before, plugging the solutions for $\Delta\mathbf{p}^T $ and $\Delta\mathbf{q}^T)^T$ into Equation \ref{eq:bidirectional_structure} we can obtain the optimal value for $\Delta\mathbf{c}^T$ using:
\begin{equation}
    \begin{aligned}
        \Delta\mathbf{c}^*_k & = \mathbf{A}^T \left( \mathbf{r} - \mathbf{J}_{\mathbf{i}} \Delta\mathbf{p}^*_{k-1} - \mathbf{J}_{\mathbf{a}} \Delta\mathbf{q}^*_{k-1} \right) 
    \label{eq:bidirectional_schur_solution2}
    \end{aligned}
\end{equation}

The Schur complement can be re-applied to Equation \ref{eq:bidirectional_schur_solution1} to derive a solution for $\Delta\mathbf{q}$ that further reduces the complexity of inverting the Hessian to $O(n^3)$:
\begin{equation}
    \begin{aligned}
        \left( \mathbf{J}_{\mathbf{a}}^T\mathbf{P}\mathbf{J}_{\mathbf{a}} \right) \Delta \mathbf{q} & = \mathbf{J}_{\mathbf{a}}^T\mathbf{P}\mathbf{r}_b
        \\
        \Delta \mathbf{q} & = \left( \mathbf{J}_{\mathbf{a}}^T\mathbf{P}\mathbf{J}_{\mathbf{a}} \right)^{-1} \mathbf{J}_{\mathbf{a}}^T\mathbf{P}\mathbf{r}_b
    \label{eq:bidirectional_schur_solution3}
    \end{aligned}
\end{equation}
where we have defined the projection matrix $\mathbf{P}$ as:
\begin{equation}
    \begin{aligned}
		\mathbf{P} &= \bar{\mathbf{A}} - \bar{\mathbf{A}}\mathbf{J}_{\mathbf{i}}^T \left( \mathbf{J}_{\mathbf{i}}^T\bar{\mathbf{A}}\mathbf{J}_{\mathbf{i}}^T \right)^{-1} \mathbf{J}_{\mathbf{i}}^T\bar{\mathbf{A}}
		\label{eq:bidirectional_schur_projection}
    \end{aligned}
\end{equation}
and the solutions for $\Delta\mathbf{q}$ and $\Delta\mathbf{c}$ can be obtained by plugging the solutions for $\Delta\mathbf{q}$ into Equation \ref{eq:bidirectional_schur_solution1} and the solutions for $\Delta\mathbf{q}$ and $\Delta\mathbf{p}$ into Equation \ref{eq:bidirectional_structure} respectively:
\begin{equation}
    \begin{aligned}
        \Delta \mathbf{p} & = \left( \mathbf{J}_{\mathbf{i}}^T\bar{\mathbf{A}}\mathbf{J}_{\mathbf{i}} \right)^{-1} \mathbf{J}_{\mathbf{i}}^T\bar{\mathbf{A}} \left(\mathbf{r}_b - \mathbf{J}_{\mathbf{a}}\Delta \mathbf{q} \right)
        \\
        \Delta\mathbf{c}^*_k & = \mathbf{A}^T \left( \mathbf{r} - \mathbf{J}_{\mathbf{i}} \Delta\mathbf{p}^*_{k-1} - \mathbf{J}_{\mathbf{a}} \Delta\mathbf{q}^*_{k-1} \right) 
    \label{eq:bidirectional_schur_solution4}
    \end{aligned}
\end{equation}

\subsubsection*{Alternated}
\label{sec:gauss_newton_alternated}

Instead of solving the problem simultaneously with respect to all parameters, we can update one set of parameters at a time while keeping the other sets fixed. In the optimization literature, this procedure is known as alternated optimization \cite{DelaTorre2012}.

More specifically, using \emph{asymmetric} composition we can alternate between updating $\Delta\mathbf{c}^*_k$ given the previous $\Delta\mathbf{p}_{k-1}^*$ and then update $\Delta\mathbf{p}^*_k$ given the updated $\Delta\mathbf{c}^*$ in an alternate manner. Taking advantage of the structure of the problem defined by Equation \ref{eq:asymmetric_structure}, we can obtain the following system of equations:
\begin{equation}
    \begin{aligned}
        \Delta\mathbf{c} + \mathbf{A}^T \mathbf{J}_{\mathbf{t}} \Delta\mathbf{p} & = \mathbf{A}^T \mathbf{r}_a 
        \\
        \mathbf{J}_{\mathbf{t}}^T \mathbf{A} \Delta\mathbf{c} + \mathbf{J}_{\mathbf{t}}^T \mathbf{J}_{\mathbf{t}} \Delta\mathbf{p} & = \mathbf{J}_a^T \mathbf{r}_a 
    \label{eq:asymmetric_alt_system}
    \end{aligned}
\end{equation}
which we can rewrite as:
\begin{equation}
    \begin{aligned}
        \Delta\mathbf{c}^*_k & = \mathbf{A}^T \left( \mathbf{r}_a - \mathbf{J}_{\mathbf{t}} \Delta\mathbf{p}^*_{k-1} \right) 
        \\
        \Delta\mathbf{p}^*_k & = \left(\mathbf{J}_{\mathbf{t}}^T\mathbf{J}_{\mathbf{t}}\right)^{-1} \mathbf{J}_{\mathbf{t}}^T \left( \mathbf{r}_a - \mathbf{A} \Delta\mathbf{c}^*_{k} \right)
        \label{eq:asymmetric_alt_solution}
    \end{aligned}
\end{equation}
in order to obtain the analytical expression for the previous alternated update rules.

In the case of \emph{bidirectional} composition we can proceed in two different ways:  
\begin{inparaenum}[\itshape a\upshape)]
\item first update $\Delta\mathbf{c}^*_k$ given $\Delta\mathbf{p}_{k-1}^*$ and $\Delta\mathbf{q}_{k-1}^*$ and then update $(\Delta\mathbf{p}_{k}^{*T}, \Delta\mathbf{q}_{k-1}^{*T})$ from the updated $\Delta\mathbf{c}^*_k$, or
\item update $\Delta\mathbf{c}^*_k$ given $\Delta\mathbf{p}_{k-1}^*$ and $\Delta\mathbf{q}_{k-1}^*$, then  $\Delta\mathbf{p}^*_k$ given the updated $\Delta\mathbf{c}^*$ and the previous $\Delta\mathbf{q}^*_{k-1}$ and, finally, $\Delta\mathbf{q}^*_k$ given $\Delta\mathbf{c}_{k}^*$ and the new $\Delta\mathbf{p}_{k}$.
\end{inparaenum}  

Given Equation \ref{eq:bidirectional_structure}, we can derive the following system of equations:
\begin{equation}
    \begin{aligned}
        \Delta\mathbf{c} + \mathbf{A}^T \mathbf{J}_\mathbf{i} \Delta\mathbf{p} + \mathbf{A}^T \mathbf{J}_\mathbf{a} \Delta\mathbf{q} & = \mathbf{A}^T \mathbf{r}_b 
        \\
        \mathbf{J}_{\mathbf{i}}^T \mathbf{A} \Delta\mathbf{c} + \mathbf{J}_{\mathbf{i}}^T \mathbf{J}_\mathbf{i} \Delta\mathbf{p} + \mathbf{J}_{\mathbf{i}}^T \mathbf{J}_\mathbf{a} \Delta\mathbf{q} & = \mathbf{J}_{\mathbf{i}}^T \mathbf{r}_b 
        \\
        \mathbf{J}_{\mathbf{a}}^T \mathbf{A} \Delta\mathbf{c} + \mathbf{J}_{\mathbf{a}}^T \mathbf{J}_\mathbf{i} \Delta\mathbf{p} + \mathbf{J}_{\mathbf{a}}^T \mathbf{J}_\mathbf{a} \Delta\mathbf{q} & = \mathbf{J}_{\mathbf{a}}^T \mathbf{r}_b 
    \label{eq:bidirectional_alt_system}
    \end{aligned}
\end{equation}
from which we can define the alternated update rules for the first of the previous two options:
\begin{equation}
    \begin{aligned}
        \Delta\mathbf{c}^*_k & = \mathbf{A}^T \left( \mathbf{r}_b - \mathbf{J}_{\mathbf{i}} \Delta\mathbf{p}^*_{k-1} - \mathbf{J}_{\mathbf{a}} \Delta\mathbf{q}^*_{k-1} \right) 
        \\
        \begin{pmatrix}
            \Delta\mathbf{p}^{*}_k
            \\
            \Delta\mathbf{q}^{*}_k
        \end{pmatrix} & = 
        \begin{pmatrix}
            \mathbf{J}_{\mathbf{i}}^T \mathbf{J}_{\mathbf{i}} & \mathbf{J}_{\mathbf{i}}^T \mathbf{J}_{\mathbf{a}}
            \\ 
            \mathbf{J}_{\mathbf{a}}^T \mathbf{J}_{\mathbf{i}} & \mathbf{J}_{\mathbf{a}}^T \mathbf{J}_{\mathbf{a}}
            \\
        \end{pmatrix}^{-1}
        \\
        & \quad \,
        \begin{pmatrix}
            \mathbf{J}_{\mathbf{i}}^T
            \\
            \mathbf{J}_{\mathbf{a}}^T
        \end{pmatrix} \left(\mathbf{r}_b - \mathbf{A} \Delta\mathbf{c}^*_{k}\right) 
        \label{eq:bidirectional_alt_solution}
    \end{aligned}
\end{equation}
and for the second:
\begin{equation}
    \begin{aligned}
        \Delta\mathbf{c}^*_k & = \mathbf{A}^T \left( \mathbf{r}_b - \mathbf{J}_{\mathbf{i}} \Delta\mathbf{p}^*_{k-1} - \mathbf{J}_{\mathbf{a}} \Delta\mathbf{q}^*_{k-1} \right) 
        \\
        \Delta\mathbf{p}^*_k & = (\mathbf{J}_{\mathbf{i}}^T\mathbf{J}_{\mathbf{i}})^{-1} \mathbf{J}_{\mathbf{i}}^T \left( \mathbf{r}_b - \mathbf{A} \Delta\mathbf{c}^*_{k} - \mathbf{J}_{\mathbf{a}} \Delta\mathbf{q}^*_{k-1} \right)
        \\
        \Delta\mathbf{q}^*_k & = (\mathbf{J}_{\mathbf{a}}^T\mathbf{J}_{\mathbf{a}})^{-1} \mathbf{J}_{\mathbf{a}}^T \left( \mathbf{r}_b - \mathbf{A} \Delta\mathbf{c}^*_{k} - \mathbf{J}_{\mathbf{i}} \Delta\mathbf{p}^*_{k} \right)
        \label{eq:bidirectional_alt_solution}
    \end{aligned}
\end{equation}

Note that the previous alternated update rules for asymmetric, Equation \ref{eq:asymmetric_alt_solution}, and bidirectional, Equation \ref{eq:bidirectional_alt_solution}, compositions are slightly different from their respective simultaneous solutions in Equations \ref{eq:asymmetric_schur_solution1} and \ref{eq:asymmetric_schur_solution2} and Equations \ref{eq:bidirectional_schur_solution1} and \ref{eq:bidirectional_schur_solution2}.

\subsubsection{Newton}
\label{sec:newton}

The Newton algorithm performs a second order Taylor expansion of the entire data term $\mathcal{D}$:
\begin{equation}
    \begin{aligned}
		\mathcal{D}(\Delta \boldsymbol{\ell}) & \approx \hat{\mathcal{D}}(\Delta \boldsymbol{\ell})
		\\
		& \approx \mathcal{D} + \frac{\partial \mathcal{D}}{\partial \Delta \boldsymbol{\ell}} \Delta \boldsymbol{\ell} + \Delta \boldsymbol{\ell}^T \frac{\partial^2 \mathcal{D}}{\partial^2 \Delta \boldsymbol{\ell}} \Delta \boldsymbol{\ell}
    \label{eq:newton_taylor}
    \end{aligned}
\end{equation}
and solves the approximate problem:
\begin{equation}
    \begin{aligned}
        \Delta \boldsymbol{\ell}^* & = \underset{\Delta \boldsymbol{\ell}}{\mathrm{arg\,min\;}} \hat{\mathcal{D}}
    \label{eq:newton_optimization_problem}
    \end{aligned}
\end{equation}

Assuming \emph{asymmetric} composition, the previous data term is define as:
\begin{equation}
    \begin{aligned}
		\mathcal{D}_a(\Delta \boldsymbol{\ell}) & = \mathbf{r}_a^T \mathbf{r}_a
    \label{eq:asymmetric_data}
    \end{aligned}
\end{equation}
and the matrix containing the first order partial derivatives with respect to the parameters (\emph{Jacobian}) can be written as:
\begin{equation}
    \begin{aligned}
		\frac{\partial \mathcal{D}_a}{\partial \Delta \boldsymbol{\ell}} & = \left( \frac{\partial \mathcal{D}_a}{\partial \Delta \mathbf{c}}, \frac{\partial \mathcal{D}_a}{\partial \Delta \mathbf{p}} \right)
		\\
		& = \left( \mathbf{A}^T \mathbf{r}_a, \mathbf{J}_{\mathbf{t}}^T \mathbf{r}_a \right)
    \label{eq:asymmetric_newton_jacobian}
    \end{aligned}
\end{equation}
On the other hand, the matrix $\frac{\partial^2 \mathcal{D}_a}{\partial^2 \Delta \boldsymbol{\ell}}$ of the second order partial derivatives (\emph{Hessian}) takes the following form: 
\begin{equation}
    \begin{aligned}
		\frac{\partial^2 \mathcal{D}_a}{\partial^2 \Delta \boldsymbol{\ell}} & = 
		\begin{pmatrix}
			\frac{\partial^2 \mathcal{D}_a}{\partial^2 \Delta \mathbf{c}} & \frac{\partial^2 \mathcal{D}_a}{\partial \Delta \mathbf{c} \partial \Delta \mathbf{p}}
			\\
			\frac{\partial^2 \mathcal{D}_a}{\partial \Delta \mathbf{p} \partial \Delta \mathbf{c}} & \frac{\partial^2 \mathcal{D}_a}{\partial^2 \Delta \mathbf{p}}
		\end{pmatrix}
    \label{eq:asymmetric_newton_hessian}
    \end{aligned}
\end{equation}
Note that the Hessian matrix is, by definition, symmetric and its individual terms are defined as follows:
\begin{equation}
    \begin{aligned}
		\frac{\partial^2 \mathcal{D}_a}{\partial^2 \Delta \mathbf{c}} & = \frac{\partial \mathbf{A}^T \mathbf{r}_a}{\partial \Delta \mathbf{c}} 
		\\
		& = \mathbf{A}^T \frac{\partial \mathbf{r}_a}{\partial \Delta \mathbf{c}} 
		\\
		& = \underbrace{\mathbf{A}^T \mathbf{A}}_{\mathbf{I}}
    \label{eq:asymmetric_hessian_term1}
    \end{aligned}
\end{equation}
\begin{equation}
    \begin{aligned}
		\frac{\partial^2 \mathcal{D}_a}{\partial \Delta \mathbf{c} \partial \Delta \mathbf{p}} & = \frac{\partial \mathbf{A}^T \mathbf{r}_a}{\partial \Delta \mathbf{p}}
		\\
		& = \frac{\partial \mathbf{A}^T}{\partial \Delta \mathbf{p}} \mathbf{r}_a + \mathbf{A}^T \frac{\partial\mathbf{r}_a}{\partial \Delta \mathbf{p}}
		\\
		& = \beta \mathbf{J}_\mathbf{A}^T \mathbf{r}_a +  \mathbf{A}^T \mathbf{J}_{\mathbf{t}}
		\\
		& = \beta \mathbf{J}_\mathbf{A}^T \mathbf{r}_a +  \mathbf{A}^T \mathbf{J}_{\mathbf{t}}
    \label{eq:asymmetric_hessian_term2}
    \end{aligned}
\end{equation}
where we have defined $\mathbf{J}_\mathbf{A} = [\nabla \mathbf{a}_1, \cdots, \nabla \mathbf{a}_m]^T \frac{\partial\mathcal{W}}{\partial \Delta \mathbf{p}}$.
\begin{equation}
    \begin{aligned}
		\frac{\partial^2 \mathcal{D}_a}{\partial^2 \Delta \mathbf{p}} & =  \frac{\partial \mathbf{J}_{\mathbf{t}}^T \mathbf{r}_a}{\partial \Delta \mathbf{p}} 
		\\
		& = \frac{\partial \mathbf{J}_{\mathbf{t}}^T}{\partial \Delta \mathbf{p}} \mathbf{r}_a + \mathbf{J}_{\mathbf{t}}^T \frac{\partial \mathbf{r}_a}{\partial \Delta \mathbf{p}} 
		\\
		& = \left( \frac{\partial\mathcal{W}}{\partial \Delta \mathbf{p}}^T \nabla^2 \mathbf{t} \frac{\partial\mathcal{W}}{\partial \Delta \mathbf{p}} + \underbrace{\nabla \mathbf{t} \overbrace{\frac{\partial^2 \mathcal{W}}{\partial^2 \mathbf{p}}}^{\mathbf{0}}}_{\mathbf{0}} \right) \mathbf{r}_a +
		\\
		& \qquad \, \mathbf{J}_{\mathbf{t}}^T \mathbf{J}_{\mathbf{t}}
		\\
		& = \left( \frac{\partial\mathcal{W}}{\partial \Delta \mathbf{p}}^T \nabla^2 \mathbf{t} \frac{\partial\mathcal{W}}{\partial \Delta \mathbf{p}} \right) \mathbf{r}_a + \mathbf{J}_{\mathbf{t}}^T \mathbf{J}_{\mathbf{t}} 
    \label{eq:asymmetric_hessian_term3}
    \end{aligned}
\end{equation}

A similar derivation can be obtained for \emph{bidirectional} composition where, as expected, the data term is define as:
\begin{equation}
    \begin{aligned}
		\mathcal{D}_b(\Delta \boldsymbol{\ell}) & = \mathbf{r}_b^T \mathbf{r}_b
    \label{eq:bidirectional_data}
    \end{aligned}
\end{equation}
In this case, the Jacobian matrix writes:
\begin{equation}
    \begin{aligned}
		\frac{\partial \mathcal{D}_b}{\partial \Delta \boldsymbol{\ell}} & = \left( \frac{\partial \mathcal{D}_b}{\partial \Delta \mathbf{c}}, \frac{\partial \mathcal{D}_b}{\partial \Delta \mathbf{p}}, \frac{\partial \mathcal{D}_b}{\partial \Delta \mathbf{q}} \right)
		\\
		& = \left( \mathbf{A}^T \mathbf{r}_a, \mathbf{J}_{\mathbf{i}}^T \mathbf{r}_a, \mathbf{J}_{\mathbf{a}}^T \mathbf{r}_a \right)
    \label{eq:bidirectional_newton_jacobian}
    \end{aligned}
\end{equation}
and the Hessian matrix takes the following form: 
\begin{equation}
    \begin{aligned}
		\frac{\partial^2 \mathcal{D}_b}{\partial^2 \Delta \boldsymbol{\ell}} & = 
		\begin{pmatrix}
			\frac{\partial^2 \mathcal{D}_b}{\partial^2 \Delta \mathbf{c}} & \frac{\partial^2 \mathcal{D}_b}{\partial \Delta \mathbf{c} \partial \Delta \mathbf{p}} & \frac{\partial^2 \mathcal{D}_b}{\partial \Delta \mathbf{c} \partial \Delta \mathbf{q}}
			\\
			\frac{\partial^2 \mathcal{D}_b}{\partial \Delta \mathbf{p} \partial \Delta \mathbf{c}} & \frac{\partial^2 \mathcal{D}_b}{\partial^2 \Delta \mathbf{p}} & \frac{\partial^2 \mathcal{D}_b}{\partial \Delta \mathbf{p} \partial \Delta \mathbf{q}}
			\\
			\frac{\partial^2 \mathcal{D}_b}{\partial \Delta \mathbf{q} \partial \Delta \mathbf{c}} & \frac{\partial^2 \mathcal{D}_b}{\partial \Delta \mathbf{q} \partial \Delta \mathbf{q}}
			& \frac{\partial^2 \mathcal{D}_b}{\partial^2 \Delta \mathbf{q}}
		\end{pmatrix}
    \label{eq:bidirectional_newton_hessian}
    \end{aligned}
\end{equation}
As before, the previous matrix is symmetric and its individual terms are defined as follows:
\begin{equation}
    \begin{aligned}
		\frac{\partial^2 \mathcal{D}_b}{\partial^2 \Delta \mathbf{c}} & = \frac{\partial \mathbf{A}^T \mathbf{r}_b}{\partial \Delta \mathbf{c}} 
		\\
		& = \mathbf{A}^T \frac{\partial \mathbf{r}_b}{\partial \Delta \mathbf{c}} 
		\\
		& = \underbrace{\mathbf{A}^T \mathbf{A}}_{\mathbf{I}}
    \label{eq:bidirectional_hessian_term1}
    \end{aligned}
\end{equation}
\begin{equation}
    \begin{aligned}
		\frac{\partial^2 \mathcal{D}_b}{\partial \Delta \mathbf{c} \partial \Delta \mathbf{p}} & = \frac{\partial \mathbf{A}^T \mathbf{r}_b}{\partial \Delta \mathbf{p}} 
		\\
		& = \mathbf{A}^T \frac{\partial \mathbf{r}_b}{\partial \Delta \mathbf{p}} 
		\\
		& = \mathbf{A}^T \mathbf{J}_{\mathbf{i}}
    \label{eq:bidirectional_hessian_term2}
    \end{aligned}
\end{equation}
\begin{equation}
    \begin{aligned}
		\frac{\partial^2 \mathcal{D}_b}{\partial \Delta \mathbf{c} \partial \Delta \mathbf{q}} & =  \frac{\partial \mathbf{A}^T \mathbf{r}_b}{\partial \Delta \mathbf{q}} 
		\\
		&= \frac{\partial \mathbf{A}^T}{\partial \Delta \mathbf{q}} \mathbf{r}_b + \mathbf{A}^T \frac{\partial \mathbf{r}_b}{\partial \Delta \mathbf{q}} 
		\\
		& = \mathbf{J}_{\mathbf{A}}^T \mathbf{r}_b + \mathbf{A}^T \mathbf{J}_{\mathbf{a}}
    \label{eq:bidirectional_hessian_term3}
    \end{aligned}
\end{equation}
\begin{equation}
    \begin{aligned}
		\frac{\partial^2 \mathcal{D}_b}{\partial^2 \Delta \mathbf{p}} & =  \frac{\partial \mathbf{J}_{\mathbf{i}}^T \mathbf{r}_b}{\partial \Delta \mathbf{p}} = \frac{\partial \mathbf{J}_{\mathbf{i}}^T}{\partial \Delta \mathbf{q}} \mathbf{r}_b + \mathbf{J}_{\mathbf{i}}^T \frac{\partial \mathbf{r}_b}{\partial \Delta \mathbf{q}} 
		\\
		& = \left( \frac{\partial\mathcal{W}}{\partial \Delta \mathbf{p}}^T \nabla^2 \mathbf{i}[\mathbf{p}] \frac{\partial\mathcal{W}}{\partial \Delta \mathbf{p}} \right) \mathbf{r}_b + \mathbf{J}_{\mathbf{i}}^T \mathbf{J}_{\mathbf{i}}
    \label{q:bidirectional_hessian_term4}
    \end{aligned}
\end{equation}
\begin{equation}
    \begin{aligned}
		\frac{\partial^2 \mathcal{D}_b}{\partial \Delta \mathbf{p} \partial \Delta \mathbf{q}} & =  \frac{\partial \mathbf{J}_{\mathbf{i}}^T \mathbf{r}_b}{\partial \Delta \mathbf{q}}
		\\
		& = \mathbf{J}_{\mathbf{i}}^T \mathbf{J}_{\mathbf{a}}
    \label{eq:bidirectional_hessian_term5}
    \end{aligned}
\end{equation}
\begin{equation}
    \begin{aligned}
		\frac{\partial^2 \mathcal{D}_b}{\partial^2 \Delta \mathbf{q}} & =  \frac{\partial \mathbf{J}_{\mathbf{a}}^T \mathbf{r}_b}{\partial \Delta \mathbf{q}}
		\\
		& = \left( \frac{\partial\mathcal{W}}{\partial \Delta \mathbf{q}}^T \nabla^2 (\mathbf{a} + \mathbf{A}\mathbf{c}) \frac{\partial\mathcal{W}}{\partial \Delta \mathbf{q}} \right) \mathbf{r}_b + \mathbf{J}_{\mathbf{a}}^T \mathbf{J}_{\mathbf{a}}
    \label{q:bidirectional_hessian_term6}
    \end{aligned}
\end{equation}


\subsubsection*{Simultaneous}
\label{sec:newton_simultaneous}

Using the Newton method we can solve for all parameters simultaneously by equating the partial derivative of Equation \ref{eq:newton_optimization_problem} to $0$:
\begin{equation}
    \begin{aligned}
    	0 & = \frac{\partial\hat{\mathcal{D}}}{\partial \Delta \boldsymbol{\ell}}
    	\\
		& = \frac{\partial \mathcal{D}}{\partial \Delta \boldsymbol{\ell}} + \frac{\partial^2 \mathcal{D}}{\partial^2 \Delta \boldsymbol{\ell}} \Delta \boldsymbol{\ell}
    \label{eq:ssd_bc}
    \end{aligned}
\end{equation}
with the solution for the parameters given by:
\begin{equation}
    \begin{aligned}
    	\Delta \boldsymbol{\ell} & = \frac{\partial^2 \mathcal{D}}{\partial^2 \Delta \boldsymbol{\ell}}^{-1} \frac{\partial \mathcal{D}}{\partial \Delta \boldsymbol{\ell}}
    \label{eq:ssd_bc}
    \end{aligned}
\end{equation}

Note that, similar to the Gauss-Newton method, the complexity of inverting the Hessian matrix $\frac{\partial^2 \mathcal{D}}{\partial^2 \Delta \boldsymbol{\ell}}$ is $O((m+2n)^3)$ and, as noted by Kossaifi et al. in \cite{Kossaifi2014}\footnote{Again, notice that, in \cite{Kossaifi2014}, Kossaifi et al. applied the Schur complement using inverse composition while we apply it here using the more general \emph{asymmetric} (which includes \emph{forward}, \emph{inverse} and \emph{symmetric}) and \emph{bidirectional} compositions.}, we can take advantage of the structure of the Hessian in Equations \ref{eq:asymmetric_newton_hessian} and \ref{eq:bidirectional_newton_hessian} to apply the Schur complement and obtain a faster algorithms. 

The solutions for $\Delta\mathbf{p}$ and $\Delta\mathbf{c}$ using \emph{asymmetric} composition are given by the following expressions:
\begin{equation}
    \begin{aligned}
        \Delta \mathbf{p} & = \left(\frac{\partial^2 \mathcal{D}_a}{\partial^2 \Delta \mathbf{p}} - \frac{\partial^2 \mathcal{D}_a}{\partial \Delta \mathbf{p} \Delta \mathbf{c}}^T \frac{\partial^2 \mathcal{D}_a}{\partial \Delta \mathbf{c} \Delta \mathbf{p}} \right)^{-1} 
        \\ 
        & \quad \, \left( \frac{\partial \mathcal{D}_a}{\partial \Delta \mathbf{p}} - \frac{\partial^2 \mathcal{D}_a}{\partial \Delta \mathbf{p} \Delta \mathbf{c}} \frac{\partial \mathcal{D}_a}{\partial \Delta \mathbf{c}} \right)
        % & = \left( \mathbf{H}_{\mathbf{t}} - \left( \beta\mathbf{J}^T_{\mathbf{A}}\mathbf{r}_a + \mathbf{A}^T\mathbf{J}_{\mathbf{t}} \right)^T \left( \beta\mathbf{J}^T_{\mathbf{A}}\mathbf{r}_a + \mathbf{A}^T\mathbf{J}_{\mathbf{t}} \right) \right)^{-1}
        % \\ 
        % & \quad \, \left( \mathbf{J}_t \bar{\mathbf{A}} \mathbf{r}_a - \beta \mathbf{r}_a^T\mathbf{J}_{\mathbf{A}}\mathbf{A}^T\mathbf{r}_a \right)
    	\\
        \Delta \mathbf{c} & = \frac{\partial \mathcal{D}_a}{\partial \Delta \mathbf{c}} - \frac{\partial^2 \mathcal{D}_a}{\partial \Delta \mathbf{c} \partial \Delta \mathbf{p}} \Delta \mathbf{p}
        % \\
        % & = \mathbf{A}^T \left( \left(\mathbf{I} - \beta \mathbf{J}_{\mathbf{A}} \right) \mathbf{r}_a - \mathbf{J}_{\mathbf{t}} \Delta \mathbf{p} \right)
    \label{eq:asymmetric_newton_schur_solutions}
    \end{aligned}
\end{equation}

On the other hand, the solutions for \emph{bidirectional} composition are given either by:
\begin{equation}
    \begin{aligned}
        \begin{pmatrix}
            \Delta\mathbf{p}
            \\
            \Delta\mathbf{q}
        \end{pmatrix} & = 
        \begin{pmatrix}
            \mathbf{V} & \mathbf{W}^T
            \\ 
            \mathbf{W} & \mathbf{U}
            \\
        \end{pmatrix}^{-1}
        \begin{pmatrix}
            \mathbf{v}
            \\
            \mathbf{u}
        \end{pmatrix}
        \\
        \Delta \mathbf{c} & = \frac{\partial \mathcal{D}_b}{\partial \Delta \mathbf{c}} - \frac{\partial^2 \mathcal{D}_b}{\partial \Delta \mathbf{c} \partial \Delta \mathbf{p}} \Delta \mathbf{p} - \frac{\partial^2 \mathcal{D}_b}{\partial \Delta \mathbf{c} \partial \Delta \mathbf{q}} \Delta \mathbf{q}
    \label{eq:bidirectional_newton_schur_solutions1}
    \end{aligned}
\end{equation}
or
\begin{equation}
    \begin{aligned}
        \Delta \mathbf{p} & = \left( \mathbf{U} - \mathbf{W} \mathbf{V}^{-1} \mathbf{W}^T \right)^{-1} \left(\mathbf{u} -  \mathbf{V} \mathbf{V}^{-1}\mathbf{v} \right)
    	\\
        \Delta \mathbf{p} & = \mathbf{V}^{-1} \left( \mathbf{v} - \mathbf{W}^T \Delta\mathbf{q}\right)
    	\\
        \Delta \mathbf{c} & = \frac{\partial \mathcal{D}_b}{\partial \Delta \mathbf{c}} - \frac{\partial^2 \mathcal{D}_b}{\partial \Delta \mathbf{c} \partial \Delta \mathbf{p}} \Delta \mathbf{p} - \frac{\partial^2 \mathcal{D}_b}{\partial \Delta \mathbf{c} \partial \Delta \mathbf{q}} \Delta \mathbf{q}
    \label{eq:bidirectional_newton_schur_solutions2}
    \end{aligned}
\end{equation}
where we have defined the following auxiliary matrices
\begin{equation}
    \begin{aligned}
    	\mathbf{U} & = \frac{\partial^2 \mathcal{D}_b}{\partial^2 \Delta \mathbf{q}} - \frac{\partial^2 \mathcal{D}_b}{\partial \Delta \mathbf{q} \Delta \mathbf{c}} \frac{\partial^2 \mathcal{D}_b}{\partial \Delta \mathbf{c} \Delta \mathbf{q}}
    	\\
    	\mathbf{W} & = \frac{\partial^2 \mathcal{D}_b}{\partial \Delta \mathbf{q} \partial \Delta \mathbf{p}} - \frac{\partial^2 \mathcal{D}_b}{\partial \Delta \mathbf{q} \Delta \mathbf{c}} \frac{\partial^2 \mathcal{D}_b}{\partial \Delta \mathbf{c} \Delta \mathbf{p}}
    	\\
    	\mathbf{V} & = \frac{\partial^2 \mathcal{D}_b}{\partial^2 \Delta \mathbf{p}} - \frac{\partial^2 \mathcal{D}_b}{\partial \Delta \mathbf{p} \Delta \mathbf{c}} \frac{\partial^2 \mathcal{D}_b}{\partial \Delta \mathbf{c} \Delta \mathbf{p}}
    \label{eq:auxiliar_matrixes}
    \end{aligned}
\end{equation}
and vectors
\begin{equation}
    \begin{aligned}
    	\mathbf{u} & = \frac{\partial \mathcal{D}_b}{\partial^2 \Delta \mathbf{q}} - \frac{\partial^2 \mathcal{D}_b}{\partial \Delta \mathbf{q} \Delta \mathbf{c}} \frac{\partial \mathcal{D}_b}{\partial \Delta \mathbf{c}}
    	\\
    	\mathbf{v} & = \frac{\partial \mathcal{D}_b}{\partial \Delta \mathbf{p}} - \frac{\partial^2 \mathcal{D}_b}{\partial \Delta \mathbf{p} \Delta \mathbf{c}} \frac{\partial \mathcal{D}_b}{\partial \Delta \mathbf{c}}
    \label{eq:auxiliar_matrixes}
    \end{aligned}
\end{equation}

The derivations of the previous solutions, for both types of composition, are analogous to the ones shown in Section \ref{sec:gauss_newton_simultaneous} for the Gauss-Newton method and, consequently, have been omitted here.

\subsubsection*{Alternated}
\label{sec:newton_alternated}

Alternated optimization rules can also be derived for the Newton algorithm following the strategy shown in Section \ref{sec:gauss_newton_alternated} for the Gauss-Newton method. Again, we simply provide the update rules for both types of composition and omit the details of their full derivation.

For \emph{asymmetric} composition the alternated rules are defined as:
\begin{equation}
    \begin{aligned}
        \Delta \mathbf{c} & = \frac{\partial \mathcal{D}_a}{\partial \Delta \mathbf{c}} - \frac{\partial^2 \mathcal{D}_a}{\partial \Delta \mathbf{c} \partial \Delta \mathbf{p}} \Delta \mathbf{p} 
        \\
        \Delta \mathbf{p} & = \frac{\partial^2 \mathcal{D}_a}{\partial^2 \Delta \mathbf{p}}^{-1} \left( \frac{\partial \mathcal{D}_a}{\partial \Delta \mathbf{p}} - \frac{\partial^2 \mathcal{D}_a}{\partial \Delta \mathbf{p} \partial \Delta \mathbf{c}} \Delta \mathbf{c} \right)
        \label{eq:asymmetric_newton_alternated_solution}
    \end{aligned}
\end{equation}

The alternated rules for \emph{bidirectional} composition case are given either by:
\begin{equation}
    \begin{aligned}
        \Delta \mathbf{c} & =  \frac{\partial \mathcal{D}_b}{\partial \Delta \mathbf{c}} - \frac{\partial^2 \mathcal{D}_b}{\partial \Delta \mathbf{c} \partial \Delta \mathbf{p}} \Delta \mathbf{p} - \frac{\partial^2 \mathcal{D}_b}{\partial \Delta \mathbf{c} \partial \Delta \mathbf{q}} \Delta \mathbf{q}
        \\
        \begin{pmatrix}
            \Delta\mathbf{p}
            \\
            \Delta\mathbf{q}
        \end{pmatrix} & = 
        \begin{pmatrix}
            \frac{\partial^2 \mathcal{D}_b}{\partial^2 \Delta \mathbf{p}} & \frac{\partial^2 \mathcal{D}_b}{\partial \Delta \mathbf{c} \partial \Delta \mathbf{p}} 
            \\ 
            \frac{\partial^2 \mathcal{D}_b}{\partial \Delta \mathbf{p} \partial \Delta \mathbf{c}} & \frac{\partial^2 \mathcal{D}_b}{\partial^2 \Delta \mathbf{p}} 
            \\
        \end{pmatrix}^{-1}
        \\
        & \quad \,
        \begin{pmatrix}
            \frac{\partial \mathcal{D}_b}{\partial \Delta \mathbf{p}} - \frac{\partial^2 \mathcal{D}_b}{\partial \Delta \mathbf{p} \partial \Delta \mathbf{c}} \Delta \mathbf{c}
            \\
            \frac{\partial \mathcal{D}_b}{\partial \Delta \mathbf{q}} - \frac{\partial^2 \mathcal{D}_b}{\partial \Delta \mathbf{q} \partial \Delta \mathbf{c}} \Delta \mathbf{c}
        \end{pmatrix}
        \label{eq:bidirectional_newton_alternated_solution1}
    \end{aligned}
\end{equation}
or:
\begin{equation}
    \begin{aligned}
        \Delta \mathbf{c} & =  \frac{\partial \mathcal{D}_b}{\partial \Delta \mathbf{c}} - \frac{\partial^2 \mathcal{D}_b}{\partial \Delta \mathbf{c} \partial \Delta \mathbf{p}} \Delta \mathbf{p} - \frac{\partial^2 \mathcal{D}_b}{\partial \Delta \mathbf{c} \partial \Delta \mathbf{q}} \Delta \mathbf{q}
        \\
        \Delta \mathbf{p} & = \frac{\partial^2 \mathcal{D}_b}{\partial^2 \Delta \mathbf{p}}^{-1} 
        \\
        & \quad \left( \frac{\partial \mathcal{D}_b}{\partial \Delta \mathbf{p}} - \frac{\partial^2 \mathcal{D}_b}{\partial \Delta \mathbf{p} \partial \Delta \mathbf{c}} \Delta \mathbf{c} - \frac{\partial^2 \mathcal{D}_b}{\partial \Delta \mathbf{p} \partial \Delta \mathbf{q}} \Delta \mathbf{q} \right)
        \\
        \Delta \mathbf{q} & = \frac{\partial^2 \mathcal{D}_b}{\partial^2 \Delta \mathbf{q}}^{-1} 
        \\
        & \quad \left( \frac{\partial \mathcal{D}_b}{\partial \Delta \mathbf{q}} - \frac{\partial^2 \mathcal{D}_b}{\partial \Delta \mathbf{q} \partial \Delta \mathbf{c}} \Delta \mathbf{c} - \frac{\partial^2 \mathcal{D}_b}{\partial \Delta \mathbf{q} \partial \Delta \mathbf{p}} \Delta \mathbf{p} \right)
        \label{eq:bidirectional_newton_alternated_solution2}
    \end{aligned}
\end{equation}

\subsubsection{Wiberg}
\label{sec:wiberg}

The idea behind the Wiberg method is similar to the one used by the alternated Gauss-Newton method, Section \ref{sec:gauss_newton_alternated}. However Wiberg does so by rewriting the asymmetric $\mathbf{r}_a(\Delta\mathbf{c}, \Delta\mathbf{p})$ and bidirectional $\mathbf{r}_b(\Delta\mathbf{c}, \Delta\mathbf{p},\Delta\mathbf{q})$ residuals as functions that only depend on $\Delta\mathbf{p}$ and $\Delta\mathbf{q}$ respectively.

For \emph{asymmetric} composition, the previous Wiberg residual $\bar{\mathbf{r}}_a (\Delta \mathbf{p})$ is defined as follows:
\begin{equation}
    \begin{aligned}
        \bar{\mathbf{r}}_a (\Delta \mathbf{p}) & = \mathbf{r}_a(\bar{\Delta \mathbf{c}}(\Delta \mathbf{p}), \Delta \mathbf{p})
        \\ 
        & = \mathbf{i}[\mathbf{p} \circ \alpha \Delta \mathbf{p}] - (\mathbf{a} + \mathbf{A}(\mathbf{c} + \bar{\Delta \mathbf{c}}_a(\Delta \mathbf{p}))) [\beta \Delta\mathbf{p}]
    \label{eq:asymmetric_wiberg_residual}
    \end{aligned}
\end{equation}
where the function $\bar{\Delta \mathbf{c}}_a(\Delta \mathbf{p})$ is obtained by solving for $\Delta\mathbf{c}$ while keeping $\Delta\mathbf{p}$ fixed:
\begin{equation}
    \begin{aligned}
        \bar{\Delta \mathbf{c}}_a(\Delta \mathbf{p}) & = \mathbf{A}^T \mathbf{r}_a
        \label{eq:asymmetric_wiberg_c_function}
    \end{aligned}
\end{equation}
Given the previous residual, the Wiberg method defines the following equivalent optimization problem:
\begin{equation}
    \begin{aligned}
        \Delta\mathbf{p}^* & = \underset{\Delta\mathbf{p}}{\mathrm{arg\,min\;}} \bar{\mathbf{r}}_a^T\bar{\mathbf{r}}_a
    \label{eq:asymmetric_wiberg_problem1}
    \end{aligned}
\end{equation}
which then solves approximately by performing a first order Taylor expansion of the new Wiberg residual $\bar{\mathbf{r}}_a (\Delta \mathbf{p})$:
\begin{equation}
    \begin{aligned}
        \Delta\mathbf{p}^* & = \underset{\Delta\mathbf{p}}{\mathrm{arg\,min\;}}  \left\| \bar{\mathbf{r}}_a(\Delta\mathbf{p}) + \frac{\partial \bar{\mathbf{r}}_a}{\partial\Delta\mathbf{p}} \Delta\mathbf{p} \right\|^2
    \label{eq:asymmetric_wiberg_problem2}
    \end{aligned}
\end{equation}
Notice that the Jacobian $\frac{\partial\bar{\mathbf{r}}}{\partial\Delta\mathbf{p}}$ is substantially different from the Gauss-Newton one in Equation \ref{eq:asymmetric_jacobian}, and it is defined as follows:
\begin{equation}
    \begin{aligned}
        \frac{d \bar{\mathbf{r}}_a}{d \Delta \mathbf{p}} & = \frac{\partial \bar{\mathbf{r}}_a}{\partial \Delta \mathbf{p}} + \frac{\partial \bar{\mathbf{r}}_a}{\partial \bar{\Delta \mathbf{c}}} \frac{\partial \bar{\Delta \mathbf{c}}}{\partial \Delta \mathbf{p}}
        \\
        & = \mathbf{J}_{\mathbf{t}} - \mathbf{A}\mathbf{A}^T\mathbf{J}_{\mathbf{t}}
        \\
        & = \bar{\mathbf{A}}\mathbf{J}_{\mathbf{t}}
    \label{eq:asymmetric_wiberg_jacobian}
    \end{aligned}
\end{equation}
 The solution for $\Delta\mathbf{p}$ is obtained as usual by equating the derivative of \ref{eq:asymmetric_wiberg_problem2} with respect to $\Delta\mathbf{p}$ to 0:
 \begin{equation}
    \begin{aligned}
    	\Delta \mathbf{p}^* & = \left( \mathbf{J}_{\mathbf{t}}^T \bar{\mathbf{A}} \mathbf{J}_{\mathbf{t}} \right)^{-1} \mathbf{J}_{\mathbf{t}}^T \bar{\mathbf{A}} \bar{\mathbf{r}}_a
    \label{eq:asymmetric_wiberg_solution}
    \end{aligned}
\end{equation}

For \emph{bidirectional} composition, the Wiberg residual $\bar{\mathbf{r}}_b (\Delta \mathbf{p})$ is defined as:
\begin{equation}
    \begin{aligned}
        \bar{\mathbf{r}}_b(\Delta\mathbf{q}) & = \mathbf{r}_b(\bar{\Delta \mathbf{c}}(\Delta\mathbf{q}), \bar{\Delta \mathbf{p}}(\bar{\Delta \mathbf{c}}(\Delta\mathbf{q}), \Delta\mathbf{q}), \Delta\mathbf{q})
        \\
        & = \mathbf{i}[\mathbf{p} \circ \bar{\Delta \mathbf{p}}(\bar{\Delta \mathbf{c}}(\Delta\mathbf{q}), \Delta\mathbf{q})] - (\mathbf{a} - \mathbf{A}(\mathbf{c} + \bar{\Delta \mathbf{c}})(\Delta\mathbf{q}))[\Delta \mathbf{q}]
    \label{eq:bidirectional_wiberg_residual}
    \end{aligned}
\end{equation}
where, similarly as before, the function $\bar{\Delta \mathbf{c}}_b(\Delta \mathbf{q})$ is defined as:
\begin{equation}
    \begin{aligned}
        \bar{\Delta \mathbf{c}}_b(\Delta \mathbf{q}) & = \mathbf{A}^T \mathbf{r}_b
        \label{eq:asymmetric_wiberg_c_function}
    \end{aligned}
\end{equation}
and the function $\bar{\Delta \mathbf{p}}(\bar{\Delta \mathbf{c}}(\Delta\mathbf{q}), \Delta\mathbf{q})$ is obtained by solving for $\Delta\mathbf{p}$ while keeping $\Delta \mathbf{q}$ fixed and $\Delta\mathbf{c}$ as given by the previous function $\bar{\Delta \mathbf{c}}_b(\Delta \mathbf{q})$:
\begin{equation}
    \begin{aligned}
        \bar{\Delta \mathbf{p}}(\bar{\Delta \mathbf{c}}(\Delta\mathbf{q}), \Delta\mathbf{q}) & = \left( \mathbf{J}_i^T \mathbf{J}_i \right)^{-1} \mathbf{J}_i^T  \mathbf{r}_b
        \label{eq:asymmetric_wiberg_p_function}
    \end{aligned}
\end{equation}
The Wiberg method proceeds to define the following equivalent optimization problem:
\begin{equation}
    \begin{aligned}
        \Delta\mathbf{q}^* & = \underset{\Delta\mathbf{q}}{\mathrm{arg\,min\;}} \bar{\mathbf{r}}_b^T\bar{\mathbf{r}}_b
    \label{eq:bidirectional_wiberg_problem1}
    \end{aligned}
\end{equation}
and then solves approximately by performing a first order Taylor expansion of the Wiberg residual $\bar{\mathbf{r}}_a (\Delta \mathbf{q})$ around $\Delta\mathbf{q}$:
\begin{equation}
    \begin{aligned}
        \Delta\mathbf{q}^* & = \underset{\Delta\mathbf{q}}{\mathrm{arg\,min\;}}  \left\| \bar{\mathbf{r}}_b(\Delta\mathbf{q}) + \frac{\partial \bar{\mathbf{r}}_b}{\partial\Delta\mathbf{q}} \Delta\mathbf{q} \right\|^2
    \label{eq:bidirectional_wiberg_problem2}
    \end{aligned}
\end{equation}
where, in this case, the Jacobian of the residual takes the following form:
\begin{equation}
    \begin{aligned}
        \frac{d \bar{\mathbf{r}}_b}{d \Delta \mathbf{q}} & = \frac{\partial \bar{\mathbf{r}}_b}{\partial \Delta \mathbf{q}} + \frac{\partial \bar{\mathbf{r}}_b}{\partial \bar{\Delta \mathbf{p}}} \frac{\partial \bar{\Delta \mathbf{p}}}{\partial \Delta \mathbf{q}} + 
        \\
        & \quad \left( \frac{\partial\bar{\mathbf{r}}_b}{\partial \bar{\Delta \mathbf{c}}} + \frac{\partial \bar{\mathbf{r}}_b}{\partial \bar{\Delta \mathbf{p}}} \frac{\partial \bar{\Delta \mathbf{p}}}{\partial \bar{\Delta \mathbf{c}}} \right) \frac{\partial \bar{\Delta \mathbf{c}}}{\partial \Delta \mathbf{q}}
        \\
        & = - \mathbf{J}_{\mathbf{a}} - \mathbf{J}_{\mathbf{i}} \left( \mathbf{J}_i^T \mathbf{J}_i \right)^{-1} \mathbf{J}_i^T \mathbf{J}_{\mathbf{a}} +
        \\
        & \quad \left(\mathbf{A} + \mathbf{J}_{\mathbf{i}} \left( \mathbf{J}_i^T \mathbf{J}_i \right)^{-1} \mathbf{J}_i^T \mathbf{A} \right) \mathbf{A}^T \mathbf{J}_{\mathbf{a}}
    \label{eq:bidirectional_wiberg_jacobian}
    \end{aligned}
\end{equation}
 The solution for $\Delta\mathbf{p}$ is obtained as usual by equating the derivative of \ref{eq:asymmetric_wiberg_problem2} with respect to $\Delta\mathbf{p}$ to 0:
 \begin{equation}
    \begin{aligned}
    	\Delta \mathbf{p}^* & = \left( \mathbf{J}_{\mathbf{t}}^T \bar{\mathbf{A}} \mathbf{J}_{\mathbf{t}} \right)^{-1} \mathbf{J}_{\mathbf{t}}^T \bar{\mathbf{A}} \bar{\mathbf{r}}_a
    \label{eq:asymmetric_wiberg_solution}
    \end{aligned}
\end{equation}





\begin{equation}
    \begin{aligned}
        \Delta\mathbf{q}^* & = \underset{\Delta\mathbf{q}}{\mathrm{arg\,min\;}} \tilde{\mathbf{r}}(\Delta\mathbf{q})^T \tilde{\mathbf{r}}(\Delta\mathbf{q}) 
        \\
        & = \underset{\Delta\mathbf{q}}{\mathrm{arg\,min\;}}  \left\| \mathbf{i}[\mathbf{p} \circ \tilde{\Delta \mathbf{p}}] - (\mathbf{a} - \mathbf{A}(\mathbf{c} + \tilde{\Delta \mathbf{c}}))[\Delta \mathbf{q}] \right\|^2
    \label{eq:ssd_appearance}
    \end{aligned}
\end{equation}

\begin{equation}
    \begin{aligned}
        \Delta\mathbf{q}^* & = \underset{\Delta\mathbf{q}}{\mathrm{arg\,min\;}}  \left\| \mathbf{r}(\Delta\mathbf{q}) + \frac{\partial\mathbf{r}}{\partial\Delta\mathbf{q}} \Delta\mathbf{q} \right\|^2
    \label{eq:ssd_appearance}
    \end{aligned}
\end{equation}

\begin{equation}
    \begin{aligned}
        \frac{d \tilde{\mathbf{r}}}{d \Delta \mathbf{q}} & = \frac{\partial \tilde{\mathbf{r}}}{\partial \Delta \mathbf{q}} + \frac{\partial \tilde{\mathbf{r}}}{\partial \tilde{\Delta \mathbf{p}}} \frac{\partial \tilde{\Delta \mathbf{p}}}{\partial \Delta \mathbf{q}} + \left( \frac{\partial\tilde{\mathbf{r}}}{\partial \tilde{\Delta \mathbf{c}}} + \frac{\partial \tilde{\mathbf{r}}}{\partial \tilde{\Delta \mathbf{p}}} \frac{\partial \tilde{\Delta \mathbf{p}}}{\partial \tilde{\Delta \mathbf{c}}} \right) \frac{\partial \tilde{\Delta \mathbf{c}}}{\partial \Delta \mathbf{q}} 
    \label{eq:wiberg_appearance}
    \end{aligned}
\end{equation}

\begin{equation}
    \begin{aligned}
        \frac{\partial \tilde{\mathbf{r}}}{\partial \Delta \mathbf{q}} &= - \mathbf{J}_{\mathbf{a}}
    \label{eq:wiberg_inverse1}
    \end{aligned}
\end{equation}

\begin{equation}
    \begin{aligned}
        \frac{\partial \tilde{\mathbf{r}}}{\partial \tilde{\Delta \mathbf{p}}} \frac{\partial \tilde{\Delta \mathbf{q}}}{\partial \Delta \mathbf{p}} & = - \mathbf{A}^T \mathbf{J}_{\mathbf{i}} \left( \mathbf{J}_{\mathbf{i}}^T \mathbf{A} \mathbf{J}_{\mathbf{i}} \right)^{-1} \mathbf{J}_{\mathbf{i}}^T \mathbf{A} \mathbf{J}_{\mathbf{a}}
    \label{eq:wiberg_inverse2}
    \end{aligned}
\end{equation}

\begin{equation}
    \begin{aligned}
        \left( \frac{\partial\tilde{\mathbf{r}}}{\partial \tilde{\Delta \mathbf{c}}} + \frac{\partial \tilde{\mathbf{r}}}{\partial \tilde{\Delta \mathbf{p}}} \frac{\partial \tilde{\Delta \mathbf{p}}}{\partial \tilde{\Delta \mathbf{c}}} \right) \frac{\partial \tilde{\Delta \mathbf{c}}}{\partial \Delta \mathbf{q}} & = \left( \mathbf{A} + \mathbf{A}^T \mathbf{J}_{\mathbf{i}} \left( \mathbf{J}_{\mathbf{i}}^T \mathbf{A} \mathbf{J}_{\mathbf{i}} \right)^{-1} \mathbf{J}_{\mathbf{i}}^T \mathbf{A} \mathbf{A}^T \right) \mathbf{A} \mathbf{J}_{\mathbf{a}}
    \label{eq:wiberg_inverse3}
    \end{aligned}
\end{equation}

\begin{equation}
    \begin{aligned}
        \frac{d \tilde{\mathbf{r}}}{d \Delta \mathbf{p}} & = \mathbf{J}_{\mathbf{i}} - \mathbf{A} \mathbf{A}^T \mathbf{J}_{\mathbf{i}}
        \\
        & = (\mathbf{I} - \mathbf{A} \mathbf{A}^T) \mathbf{J}_{\mathbf{i}}
        \\
        & = \bar{\mathbf{A}} \mathbf{J}_{\mathbf{i}}
    \label{eq:wiberg_forward}
    \end{aligned}
\end{equation}

\begin{equation}
    \begin{aligned}
        \frac{d \tilde{\mathbf{r}}}{d \Delta \mathbf{q}} & = \left( \bar{\mathbf{A}} - \bar{\mathbf{A}}\mathbf{J}_{\mathbf{i}} \left(\mathbf{J}_{\mathbf{i}}^T\bar{\mathbf{A}}\mathbf{J}_{\mathbf{i}}\right)^{-1} \mathbf{J}_{\mathbf{i}}^T \bar{\mathbf{A}}^T \right) \mathbf{J}_{\mathbf{a}}
    \label{eq:wiberg_appearance}
    \end{aligned}
\end{equation}









\begin{equation}
    \begin{aligned}
        \mathbf{r}(\Delta \mathbf{c}, \Delta \mathbf{p}, \Delta \mathbf{q}) & = \mathbf{i}[\mathbf{p} \circ \Delta \mathbf{p}] - (\mathbf{a} - \mathbf{A}(\mathbf{c} + \Delta \mathbf{c}))[\Delta \mathbf{q}]
    \label{eq:wiberg_appearance}
    \end{aligned}
\end{equation}

\begin{equation}
    \begin{aligned}
        \Delta\mathbf{c}^* & = \underset{\Delta\mathbf{c}}{\mathrm{arg\,min\;}} \mathbf{r}^T\mathbf{r}
    \label{eq:ssd_appearance}
    \end{aligned}
\end{equation}

\begin{equation}
    \begin{aligned}
        \Delta \mathbf{c}^* & = \mathbf{A}^T \mathbf{r}(\mathbf{0}, \Delta \mathbf{p}, \Delta \mathbf{q})
    \label{eq:wiberg_appearance}
    \end{aligned}
\end{equation}


\begin{equation}
    \begin{aligned}
        \tilde{\Delta \mathbf{c}}(\Delta \mathbf{p}, \Delta \mathbf{q}) & = \mathbf{A}^T \mathbf{r}(\mathbf{0}, \Delta \mathbf{p}, \Delta \mathbf{q})
    \label{eq:wiberg_appearance}
    \end{aligned}
\end{equation}

\begin{equation}
    \begin{aligned}
        \tilde{\mathbf{r}}(\tilde{\Delta \mathbf{c}}, \Delta \mathbf{p}, \Delta \mathbf{q}) & = \mathbf{i}[\mathbf{p} \circ \Delta \mathbf{p}] - (\mathbf{a} - \mathbf{A}(\mathbf{c} + \tilde{\Delta \mathbf{c}}))[\Delta \mathbf{q}]
    \label{eq:wiberg_appearance}
    \end{aligned}
\end{equation}

\begin{equation}
    \begin{aligned}
        \Delta\mathbf{p}^* & = \underset{\Delta\mathbf{c}}{\mathrm{arg\,min\;}} \tilde{\mathbf{r}}^T\tilde{\mathbf{r}}
    \label{eq:ssd_appearance}
    \end{aligned}
\end{equation}

\begin{equation}
    \begin{aligned}
        \Delta\mathbf{p}^* & = \underset{\Delta\mathbf{p}}{\mathrm{arg\,min\;}}  \left\| \tilde{\mathbf{r}}(\Delta\mathbf{p}) + \frac{\partial\tilde{\mathbf{r}}}{\partial\Delta\mathbf{p}} \Delta\mathbf{p} \right\|^2
    \label{eq:ssd_appearance}
    \end{aligned}
\end{equation}

\begin{equation}
    \begin{aligned}
        \frac{d \tilde{\mathbf{r}}}{d \Delta \mathbf{p}} & = \frac{\partial \tilde{\mathbf{r}}}{\partial \Delta \mathbf{p}} + \frac{\partial \tilde{\mathbf{r}}}{\partial \tilde{\Delta \mathbf{c}}} \frac{\partial \tilde{\Delta \mathbf{c}}}{\partial \Delta \mathbf{p}}
    \label{eq:wiberg_forward}
    \end{aligned}
\end{equation}

\begin{equation}
    \begin{aligned}
        \frac{\partial \tilde{\mathbf{r}}}{\partial \Delta \mathbf{p}} &= \mathbf{J}_{\mathbf{i}}
    \label{eq:wiberg_forward1}
    \end{aligned}
\end{equation}

\begin{equation}
    \begin{aligned}
        \frac{\partial \tilde{\mathbf{r}}}{\partial \tilde{\Delta \mathbf{c}}} \frac{\partial \tilde{\Delta \mathbf{c}}}{\partial \Delta \mathbf{p}} & = - \mathbf{A}\mathbf{A}^T\mathbf{J}_{\mathbf{i}}
    \label{eq:wiberg_forward2}
    \end{aligned}
\end{equation}

\begin{equation}
    \begin{aligned}
        \frac{d \tilde{\mathbf{r}}}{d \Delta \mathbf{p}} & = \mathbf{J}_{\mathbf{i}} - \mathbf{A} \mathbf{A}^T \mathbf{J}_{\mathbf{i}}
        \\
        & = (\mathbf{I} - \mathbf{A} \mathbf{A}^T) \mathbf{J}_{\mathbf{i}}
        \\
        & = \bar{\mathbf{A}} \mathbf{J}_{\mathbf{i}}
    \label{eq:wiberg_forward}
    \end{aligned}
\end{equation}

\begin{equation}
    \begin{aligned}
        \Delta \mathbf{p}^* & = \left( \mathbf{J}_{\mathbf{i}}^T \bar{\mathbf{A}} \mathbf{J}_{\mathbf{i}} \right)^{-1} \mathbf{J}_{\mathbf{i}}^T \bar{\mathbf{A}} \tilde{\mathbf{r}}(\Delta\mathbf{c}^*, \mathbf{0})
    \label{eq:wiberg_appearance}
    \end{aligned}
\end{equation}

\begin{equation}
    \begin{aligned}
        \tilde{\Delta \mathbf{p}}(\Delta \mathbf{q}) & = \left( \mathbf{J}_{\mathbf{i}}^T \bar{\mathbf{A}} \mathbf{J}_{\mathbf{i}} \right)^{-1} \mathbf{J}_{\mathbf{i}}^T \bar{\mathbf{A}} \tilde{\mathbf{r}}(\Delta\mathbf{c}^*, \mathbf{0})
    \label{eq:wiberg_appearance}
    \end{aligned}
\end{equation}



\begin{table*}
\begin{tabular}{lccccccc}
\hline\noalign{\smallskip}
Algorithm & $\mathbf{a}$ & $\mathbf{J}$ & $\mathbf{J}^T\mathbf{Q}$ & $\mathbf{H}$ & $\mathbf{H}^{-1}$ & $\mathbf{J}^T\mathbf{e}$ & $\mathbf{H}^{-1}\mathbf{J}^T\mathbf{e}$ \\
\noalign{\smallskip}\hline\noalign{\smallskip}

SFGs & $O(mF)$ & $O(nF)$ & - & $O((m+n)^2F)$ & $O((m+n)^3)$ & $O((m+n)F)$ & $O((m+n)^2)$ \\
SFGa & $O(mF)$ & $O(nF)$ & - & $O((m+n)^2F)$ & $O((m+n)^3)$ & $O((m+n)F)$ & $O((m+n)^2)$ \\
SFW & $O(mF)$ & $O(nF)$ & - & $O((m+n)^2F)$ & $O((m+n)^3)$ & $O((m+n)F)$ & $O((m+n)^2)$ \\

SIGs & $O(mF)$ & $O(nF)$ & - & $O((m+n)^2F)$ & $O((m+n)^3)$ & $O((m+n)F)$ & $O((m+n)^2)$ \\
SIGa & $O(mF)$ & $O(nF)$ & - & $O((m+n)^2F)$ & $O((m+n)^3)$ & $O((m+n)F)$ & $O((m+n)^2)$ \\
SIW & $O(mF)$ & $O(nF)$ & - & $O((m+n)^2F)$ & $O((m+n)^3)$ & $O((m+n)F)$ & $O((m+n)^2)$ \\

SAGs & $O(mF)$ & $O(nF)$ & - & $O((m+n)^2F)$ & $O((m+n)^3)$ & $O((m+n)F)$ & $O((m+n)^2)$ \\
SAGa & $O(mF)$ & $O(nF)$ & - & $O((m+n)^2F)$ & $O((m+n)^3)$ & $O((m+n)F)$ & $O((m+n)^2)$ \\
SAW & $O(mF)$ & $O(nF)$ & - & $O((m+n)^2F)$ & $O((m+n)^3)$ & $O((m+n)F)$ & $O((m+n)^2)$ \\

SBGs & $O(mF)$ & $O(nF)$ & - & $O((m+n)^2F)$ & $O((m+n)^3)$ & $O((m+n)F)$ & $O((m+n)^2)$ \\
SBGa & $O(mF)$ & $O(nF)$ & - & $O((m+n)^2F)$ & $O((m+n)^3)$ & $O((m+n)F)$ & $O((m+n)^2)$ \\
SBW & $O(mF)$ & $O(nF)$ & - & $O((m+n)^2F)$ & $O((m+n)^3)$ & $O((m+n)F)$ & $O((m+n)^2)$ \\

PFGs & $O(mF)$ & $O(nF)$ & - & $O((m+n)^2F)$ & $O((m+n)^3)$ & $O((m+n)F)$ & $O((m+n)^2)$ \\
PFGa & $O(mF)$ & $O(nF)$ & - & $O((m+n)^2F)$ & $O((m+n)^3)$ & $O((m+n)F)$ & $O((m+n)^2)$ \\
PFW & $O(mF)$ & $O(nF)$ & - & $O((m+n)^2F)$ & $O((m+n)^3)$ & $O((m+n)F)$ & $O((m+n)^2)$ \\

PIGs & $O(mF)$ & $O(nF)$ & - & $O((m+n)^2F)$ & $O((m+n)^3)$ & $O((m+n)F)$ & $O((m+n)^2)$ \\
PIGa & $O(mF)$ & $O(nF)$ & - & $O((m+n)^2F)$ & $O((m+n)^3)$ & $O((m+n)F)$ & $O((m+n)^2)$ \\
PIW & $O(mF)$ & $O(nF)$ & - & $O((m+n)^2F)$ & $O((m+n)^3)$ & $O((m+n)F)$ & $O((m+n)^2)$ \\

PAGs & $O(mF)$ & $O(nF)$ & - & $O((m+n)^2F)$ & $O((m+n)^3)$ & $O((m+n)F)$ & $O((m+n)^2)$ \\
PAGa & $O(mF)$ & $O(nF)$ & - & $O((m+n)^2F)$ & $O((m+n)^3)$ & $O((m+n)F)$ & $O((m+n)^2)$ \\
PAW & $O(mF)$ & $O(nF)$ & - & $O((m+n)^2F)$ & $O((m+n)^3)$ & $O((m+n)F)$ & $O((m+n)^2)$ \\

PBGs & $O(mF)$ & $O(nF)$ & - & $O((m+n)^2F)$ & $O((m+n)^3)$ & $O((m+n)F)$ & $O((m+n)^2)$ \\
PBGa & $O(mF)$ & $O(nF)$ & - & $O((m+n)^2F)$ & $O((m+n)^3)$ & $O((m+n)F)$ & $O((m+n)^2)$ \\
PBW & $O(mF)$ & $O(nF)$ & - & $O((m+n)^2F)$ & $O((m+n)^3)$ & $O((m+n)F)$ & $O((m+n)^2)$ \\

\noalign{\smallskip}\hline
\end{tabular}
\caption{Computational cost per iteration of all Compositional Gradient Descent algorithms for fitting AAMs.}
\label{tab:complexity}  
\end{table*}