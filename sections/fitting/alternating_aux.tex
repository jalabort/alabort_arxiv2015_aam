\subsubsection{Alternating}
\label{sec:alternating}

Alternating optimization algorithms solve the minimization problem posed by Eq. \ref{eq:ssd} by updating the appearance parameters $\mathbf{c}$ from the shape parameters $\mathbf{p}$ and vice versa in an alternating manner, starting from some initial values $\mathbf{p}_0$ and $\mathbf{c}_0$. 

Noting that the residual $\mathbf{r}(\mathbf{p}, \mathbf{c})$ is linear with respect to the appearance parameters $\mathbf{c}$ and assuming the shape parameters $\mathbf{p}$ to be fixed leads to a linear least squares problem with respect to $\mathbf{c}$:
\begin{equation}
    \begin{align}
        \mathbf{c}_0 & = \underset{\mathbf{c}}{\mathrm{arg\,min\;}} \left\| \mathbf{i}[\mathbf{p}_0] - \bar{\mathbf{a}} - \mathbf{A} \mathbf{c} \right\|^2 
    \label{eq:ssd_appearance}
    \end{align}
\end{equation}
the solution of which is given by:
\begin{equation}
    \begin{align}
        \mathbf{c}_0 & = (\mathbf{A}^T \mathbf{A})^{-1} \mathbf{A}^T \left( \mathbf{i}[\mathbf{p}_0] - \bar{\mathbf{a}} \right)
        \\
        & = \mathbf{A}^T \left( \mathbf{i}[\mathbf{p}_0] - \bar{\mathbf{a}} \right)
    \label{eq:solve_c0}
    \end{align}
\end{equation}
Substituting $\mathbf{c}_0$ back into Eq. \ref{eq:ssd} leads to the same minimization problem defined by Eq. \ref{eq:ssd_shape} which can be solved for a new estimate of the shape parameters $\mathbf{p}_1}$ using one iteration of the general compositional Gauss-Newton algorithm described in Sec. \ref{sec:cgd}. Given $\mathbf{p}_1$, the new estimate for the appearance parameters $\mathbf{c}_1$ can be computed by substituting it back into Eq. \ref{eq:ssd_appearance}. This alternating procedure is repeated until convergence.

To the best of our knowledge, the forward version of this algorithm, the \emph{Alternating Forward Compositional} (AFC) algorithm, has never been used in the existent literature. Similar to other forward algorithms the Jacobian $\nabla \mathbf{i} [\mathbf{p}] \frac{\partial\mathcal{W}}{\partial\mathbf{p}}$ needs to be re-computed at every iteration because of its dependency on the current estimate of the shape parameter. Its complexity per iteration is $O(n^2F + n^3)$.

The \emph{Alternating Inverse Compositional} (AIC) algorithm was first introduced by Tzimiropoulos et al. in \cite{Tzimiropoulos2012}. Note that the inverse Jacobian $\nabla \mathbf{a} \frac{\partial\mathcal{W}}{\partial\mathbf{p}}$ depends on the current estimate of the shape parameters through Eq. \ref{eq:} and, as in SIC, also needs to be recomputed at every iteration. The computational cost of each iteration of the algorithm is $O(n^2F + n^3)$.

A second family of alternating optimization algorithms can be derived by clever manipulation of the iterative SIC solution in Eq. \ref{}.
\begin{equation}
    \begin{align}
        \delta\mathbf{q} & =  \left( \mathbf{J}_\text{sim}^T\mathbf{J}_\text{sim} \right)^{-1} \mathbf{J}_\text{sim}^T \mathbf{r}
        \\
        \mathbf{J}_\text{sim}^T\mathbf{J}_\text{sim} \delta\mathbf{q} & = \mathbf{J}_\text{sim}^T \mathbf{r} 
        \\
        \begin{pmatrix}
            \mathbf{A}^T\mathbf{A} & \mathbf{A}^T\mathbf{J} 
            \\ 
            \mathbf{J}^T\mathbf{A} & \mathbf{J}^T\mathbf{J} 
        \end{pmatrix}
        \begin{pmatrix}
            \delta\mathbf{c} 
            \\ 
            \delta\mathbf{p}
        \end{pmatrix}
        & = 
        \begin{pmatrix}
            \mathbf{A}^T\mathbf{r} 
            \\ 
            \mathbf{J}^T\mathbf{r} 
        \end{pmatrix}
    \label{eq:fastsim_solution}
    \end{align}
\end{equation}
From which we can define the following system of equations:
\begin{equation}
    \begin{align}
        \mathbf{A}^T\mathbf{A} \delta\mathbf{c} + \mathbf{A}^T\mathbf{J} \delta\mathbf{p} & = \mathbf{A}^T \mathbf{r} 
        \\
        \mathbf{J}^T\mathbf{A} \delta\mathbf{c} + \mathbf{J}^T\mathbf{J} \delta\mathbf{p} & = \mathbf{J}^T \mathbf{r}
    \label{eq:fastsim_system}
    \end{align}
\end{equation}
which can be rewritten as:
\begin{equation}
    \begin{align}
        \delta\mathbf{c} & = (\mathbf{A}^T\mathbf{A})^{-1} \mathbf{A}^T \left( \mathbf{r} - \mathbf{J} \delta\mathbf{p} \right) 
        \\
        \delta\mathbf{p} & = (\mathbf{J}^T\mathbf{J})^{-1} \mathbf{J}^T \left( \mathbf{r} - \mathbf{A} \delta\mathbf{c} \right)
    \label{eq:fastsim_system}
    \end{align}
\end{equation}
