\section{Fitting Active Appearance Models}
\label{sec:fit}

AAM fitting is typically formulated as a search over the shape and appearance parameters that minimize the Sum of Squared Differences (SSD) between the warped input image $I(\mathcal{W}(\mathbf{x}; \mathbf{p}))$ and the linear appearance model $T(\mathbf{x}) + \sum_{i=1}^m \mathbf{c}_i U_i(\mathbf{x})$, i.e.: 

\begin{align}
\mathbf{p}_o, \mathbf{c}_o & = \underset{\mathbf{p}, \mathbf{c}}{\mathrm{arg\,min\;}}
\mathcal{D}(I, \mathbf{p}, \mathbf{c})
\\
\mathbf{p}_o, \mathbf{c}_o & = \underset{\mathbf{p}, \mathbf{c}}{\mathrm{arg\,min\;}} 
\sum_{\mathbf{x} \in \Omega} \left\| I(\mathcal{W}(\mathbf{x}; \mathbf{p})) - \left( T(\mathbf{x}) + \sum_{i=1}^m \mathbf{c}_i U_i(\mathbf{x}) \right) \right\|^2 
\end{align}

The previous equation can be rewritten in vector form as:

\begin{equation}
\mathbf{p}_o, \mathbf{c}_o = \underset{\mathbf{p}, \mathbf{c}}{\mathrm{arg\,min\;}} 
|| \mathbf{i}[\mathbf{p}] - (\mathbf{t} + \mathbf{U} \mathbf{c}) ||^2 
\label{eq:ssd}
\end{equation}

This vector notation will be used through out this section.

\subsection{Existing Strategies}

Several strategies have been proposed to solve the optimization problem posed by equation \ref{eq:ssd}. As previously mentioned, in this work, we limit the discussion to compositional Gradient Descent (GD) algorithms \cite{} and, consequently, we will not review discriminative and regression based approaches. For such methods the reader is referred
to the existent literature \cite{}.

Assuming, for the time being, that the optimal appearance parameters $\mathbf{c}_o$ are known, the problem reduces to an image alignment problem between the warped image $\mathbf{i}[\mathbf{p}]$ and the appearance model instance $\mathbf{t}_{\mathbf{c}_o} = \mathbf{t} + \mathbf{U} \mathbf{c}_o$

\begin{align}
\mathbf{p}_o & = \underset{\mathbf{p}}{\mathrm{arg\,min\;}} 
|| \mathbf{i}[\mathbf{p}] - \left( \mathbf{t} + \mathbf{U} \mathbf{c}_o \right) ||^2 
\\
& = \underset{\mathbf{p}}{\mathrm{arg\,min\;}} 
|| \mathbf{i}[\mathbf{p}] - \mathbf{t}_{\mathbf{c}_o} ||^2 
\label{eq:ssd_shape}
\end{align}

Compositional GD algorithms solve equation \ref{eq:ssd_shape} iteratively with respect to increments on the shape parameters by:

\begin{enumerate}

\item Introducing an incremental warp $\mathcal{W}(\mathbf{x}; \delta \mathbf{p})$

This incremental warp used to deform the image, by composing it with the current warp $\mathcal{W}(\mathbf{x}; \mathbf{p})$, in what is known as forward composition

\begin{equation}
\delta \mathbf{p}_o = \underset{\delta \mathbf{p}}{\mathrm{arg\,min\;}} 
|| \mathbf{i} [\mathbf{p} \circ \delta \mathbf{p}] - \left( \mathbf{t} + \mathbf{U} \mathbf{c}_o \right) ||^2 
\end{equation}

or directly used to deform the appearance model, in what is known as inverse composition

\begin{equation}
\delta \mathbf{p}_o = \underset{\delta \mathbf{p}}{\mathrm{arg\,min\;}} 
|| \mathbf{i} [\mathbf{p}] - \mathbf{t}[\delta \mathbf{p}] + \sum_{i=1}^m \mathbf{c}_i \mathbf{u}_i[\delta \mathbf{p}] ||^2 
\end{equation}

\item Performing a Taylor expansion around the incremental warp

\begin{equation}
\delta \mathbf{p}_o = \underset{\mathbf{p}}{\mathrm{arg\,min\;}} 
|| \mathbf{i} [\mathbf{p}] - (\mathbf{t} + \mathbf{U} \mathbf{c}) + \mathbf{J} \delta \mathbf{p} ||^2
\label{eq:sim_linear}
\end{equation}

\item Solving for $\delta \mathbf{p}$

\begin{equation}
\delta \mathbf{p} = 
\left( \mathbf{J}^T \mathbf{J} \right)^{-1} \mathbf{J}^T \left( \mathbf{i}[\mathbf{p}] - (\mathbf{t} + \mathbf{U} \mathbf{c}) \right)
\end{equation}

\item Computing the optimal shape parameters $\mathbf{p}_o$ by composing the current estimate of the warp with the incremental warp.

Forward:

\begin{equation}
\mathcal{W}(\mathbf{x}; \mathbf{p}) \leftarrow \mathcal{W}(\mathbf{x}; \mathbf{p}) \circ \mathcal{W}(\mathbf{x}; \delta \mathbf{p}) 
\end{equation}

Inverse:

\begin{equation}
\mathcal{W}(\mathbf{x}; \mathbf{p}) \leftarrow \mathcal{W}(\mathbf{x}; \mathbf{p}) \circ \mathcal{W}(\mathbf{x}; \delta \mathbf{p})^{-1} 
\end{equation}


\end{enumerate}

For further details on how to compute $\frac{\partial \mathcal{W}(\mathbf{x}; \mathbf{p})}{\partial \mathbf{p}}$ and on warp composition and inversion the reader is referred to Section \ref{sec:warp} of this paper as well as to \cite{}.

Optimization of equation \ref{eq:ssd} with respect to the appearance parameters $\mathbf{c}$ is highly dependent on the particular algorithm used. All algorithms either solve for $\mathbf{c}$ directly in close from solution, iteratively for increments $\delta \mathbf{c}$ followed by an additive update rule $\mathbf{c} \leftarrow \mathbf{c} + \delta \mathbf{c}$, or avoid solving for $\mathbf{c}$ by using projection operators.

The following subsections provide a detailed review of the different families in which existing GD algorithms can be grouped on.
The next subsections provide a detailed review of the existent families in which compositional gradient descent algorithms for fitting AAMS can be grouped on

%------------------------------------------------------------------------------------------------------------------------------------------------------------------------------------

\subsubsection{Simultaneous Family}


\paragraph{The Simultaneous Inverse Compositional (SIC) algorithm}

, first proposed in \cite{}, finds optimal shape and texture parameter increments, $\delta \mathbf{p}$ and
$\delta \mathbf{c}$, at each iteration by solving the following optimization problem

\begin{equation}
\delta \mathbf{p}_o, \delta \mathbf{c}_o = \underset{\delta \mathbf{p}, \delta \mathbf{c}}{\mathrm{arg\,min\;}} 
|| \mathbf{i} [\mathbf{p}] - \mathbf{t}[\delta \mathbf{p}] + \sum_{i=1}^m (\mathbf{c}_i + \delta \mathbf{c}_i) \mathbf{u}_i[\delta \mathbf{p}] ||^2 
\end{equation}

and uses the following update rules to obtain the optimal values for the shape and appearance parameters, $\mathbf{p}$ and $\mathbf{c}$

\begin{align}
\mathbf{p} & \leftarrow \mathbf{p} \circ \mathbf{\delta \mathbf{p}}^{-1} \\
\mathbf{c} & \leftarrow \mathbf{c} + \mathbf{\delta \mathbf{c}}
\end{align}

Linearizing the appearance model with respect to the incremental warp around $\mathbf{p} = \mathbf{0}$, leads to the following expression:

\begin{equation}
\delta \mathbf{p}_o, \delta \mathbf{c}_o = \underset{\mathbf{p}, \mathbf{c}}{\mathrm{arg\,min\;}} 
|| \mathbf{i} [\mathbf{p}] - (\mathbf{t} + \mathbf{U} (\mathbf{c} + \delta \mathbf{c})) + \mathbf{J} \delta \mathbf{p} ||^2
\label{eq:sim_linear}
\end{equation}

where the so called steepest descent images $\mathbf{J}$ are defined as

\begin{equation}
\mathbf{J} = 
\left( \nabla \mathbf{t} + \sum_{i=1}^m \mathbf{c}_i \nabla \mathbf{u}_i \right) \left. \frac{\partial \mathcal{W}(\mathbf{x}; \mathbf{p})}{\partial \mathbf{p}} \right|_{\mathbf{p}=\mathbf{0}} =
\nabla (\mathbf{t} + \mathbf{U} \mathbf{c}) \left. \frac{\partial \mathcal{W}(\mathbf{x}; \mathbf{p})}{\partial \mathbf{p}} \right|_{\mathbf{p}=\mathbf{0}}
\end{equation}

in which all second order terms of the form $\delta \mathbf{c}_i \nabla \mathbf{u}_i \left. \frac{\partial \mathcal{W}(\mathbf{x}; \mathbf{p})}{\partial \mathbf{p}} \right|_{\mathbf{p}=\mathbf{0}} \delta \mathbf{p}$ have been neglected

The solution to equation \ref{eq:sim_linear} is given by:

\begin{equation}
\begin{pmatrix}
\delta \mathbf{p}
\\ 
\delta \mathbf{c}
\end{pmatrix} 
= 
\left( \mathbf{J}_{sim}^T \mathbf{J}_{sim} \right)^{-1} \mathbf{J}_{sim}^T \left( \mathbf{i}[\mathbf{p}] - (\mathbf{t} + \mathbf{U} \mathbf{c}) \right)
\end{equation}

where the \emph{simultaneous} jacobian $\mathbf{J}_{sim} = [\mathbf{J}, \mathbf{U}]$ is given by the concatenation of the steepest descent images $\mathbf{J}$ with the appearance model bases $\mathbf{U}$.

%----------

\paragraph{The Simultaneous Forward Compositional (SFC) algorithm} 

is the analogous of the previous SIC Algorithm for the forward compositional case. The only difference between both algorithms is that in this case the incremental warp $\delta \mathbf{p}$ is introduced on the image side

\begin{equation}
\delta \mathbf{p}_o, \delta \mathbf{c}_o = \underset{\delta \mathbf{p}, \delta \mathbf{c}}{\mathrm{arg\,min\;}} 
|| \mathbf{i} [\mathbf{p} \circ \delta \mathbf{p}] - \left( \mathbf{t} + \mathbf{U} \mathbf{c} \right) ||^2
\end{equation}

Linearization of equation \ref{} leads to the previous equation \ref{} in which, in this case, the steepest descent images $\mathbf{J}$ are defined with respect to the warped image $\mathbf{i}[\mathbf{p}]$

\begin{equation}
\mathbf{J} = 
\nabla \mathbf{i}[\mathbf{p}] \left. \frac{\partial \mathcal{W}(\mathbf{x}; \mathbf{p})}{\partial \mathbf{p}} \right|_{\mathbf{p}=\mathbf{0}}
\end{equation}

Note that no second order terms need to be discarded in this case. The solutions for $\delta \mathbf{p}$ and $\delta \mathbf{c}$ are obtained from equation \ref{} and the optimal value of the parameters obtained using the following update rules

\begin{align}
\mathbf{p} & \leftarrow \mathbf{p} \circ \mathbf{\delta \mathbf{p}} \\
\mathbf{c} & \leftarrow \mathbf{c} + \mathbf{\delta \mathbf{c}}
\end{align}

To the best of our knowledge, this algorithm has not being formally proposed in the AAM literature. However, approximations of this algorithm have been extensively used in the closely related field of 3D Morphable Models \cite{} where inverse compositional algorithms cannot, in principle, be used.

%------------------------------------------------------------------------------------------------------------------------------------------------------------------------------------

\paragraph{The Fast Simultaneous Inverse Compositional (Fast-SIC) algorithm}

is a faster version of the original SIC algorithm that avoids the explicit construction and inversion of the \emph{simultaneous} hessian $\mathbf{H}_{sim} = \mathbf{J}_{sim}^T\mathbf{J}_{sim}$ was originally proposed in \cite{} and recently popularized by \cite{}.

\paragraph{The Fast Forward Compositional (Fast-SFC) algorithm}

\subsubsection{Project-Out Family}

\subsubsection{Alternating Family}

%------------------------------------------------------------------------------------------------------------------------------------------------------------------------------------

\subsection{A Probabilistic Formulation}

Given the probabilistic appearance model formulation introduced in Section \ref{sec:prob_aam}, for a particular input image we have

\begin{align}
\mathbf{i}[\mathbf{p}] & = \mathbf{m} + \mathbf{U} \mathbf{c} + \boldsymbol{\epsilon}
\\
\mathbf{c} &\approx \mathcal{N}(\mathbf{0}, \boldsymbol{\Lambda})
\\
\boldsymbol{\epsilon} & \approx \mathcal{N}(\mathbf{0}, \sigma^2 \mathbf{I})
\end{align}

%------------------------------------------------------------------------------------------------------------------------------------------------------------------------------------

\subsubsection{Maximum Likelihood}

One can easily define a Maximum Likelihood (ML) procedure to infer the optimal shape and appearance parameters given the input image

\begin{align}
\mathbf{p}_o, \mathbf{c}_o & = \underset{\mathbf{p}, \mathbf{c}}{\mathrm{arg\,max\;}} 
p(\mathbf{i}[\mathbf{p}] | \mathbf{p}, \mathbf{c}, \Theta)
\\
& = \underset{\mathbf{p}, \mathbf{c}}{\mathrm{arg\,max\;}} 
\ln p(\mathbf{i}[\mathbf{p}] | \mathbf{p}, \mathbf{c}, \Theta)
\\
& = \underset{\mathbf{p}, \mathbf{c}}{\mathrm{arg\,min\;}} 
- \ln p(\mathbf{i}[\mathbf{p}] | \mathbf{p}, \mathbf{c}, \Theta)
\\
& = \underset{\mathbf{p}, \mathbf{c}}{\mathrm{arg\,min\;}} 
\frac{1}{\sigma^2} \left\| \mathbf{i}[\mathbf{p} - \left( \mathbf{t} + \mathbf{U}\mathbf{c} \right) \right\|
\end{align}

Equation data term in equation \ref{} is the a weighted version of the data term used to derive all the algorithms reviewed in the previous subsections. The only difference is that under the proposed probabilistic AAM formulation equation equation \ref{} is weighted by the inverse of the estimated noise variance.

\subsubsection{Maximum A Posteriori}

Similarly, a Maximum A Posteriori (MAP) procedure can be obtained by taking into account the prior distribution over the shape and texture parameters

\begin{align}
\mathbf{p}_o, \mathbf{c}_o & = \underset{\mathbf{p}, \mathbf{c}}{\mathrm{arg\,max\;}} 
p(\mathbf{p}, \mathbf{c}, \mathbf{i}[\mathbf{p}] | \mathbf{p}, \mathbf{c}, \Theta)
\\
& = \underset{\mathbf{p}, \mathbf{c}}{\mathrm{arg\,max\;}} 
p(\mathbf{p}) p(\mathbf{c}) p(\mathbf{i}[\mathbf{p}] | \mathbf{p}, \mathbf{c}, \Theta)
\\
& = \underset{\mathbf{p}, \mathbf{c}}{\mathrm{arg\,max\;}} 
\ln p(\mathbf{p}) + \ln p(\mathbf{c}) + \ln p(\mathbf{i}[\mathbf{p}] | \mathbf{p}, \mathbf{c}, \Theta)
\\
& = \underset{\mathbf{p}, \mathbf{c}}{\mathrm{arg\,min\;}} 
- \ln p(\mathbf{p}) - \ln p(\mathbf{c}) - \ln p(\mathbf{i}[\mathbf{p}] | \mathbf{p}, \mathbf{c}, \Theta)
\\
& = \underset{\mathbf{p}, \mathbf{c}}{\mathrm{arg\,min\;}} 
\left\| \mathbf{p} \right\|_{\boldsymbol{\Lambda}^{-1}} + \left\| \mathbf{c} \right\|_{\boldsymbol{\Sigma}^{-1}} + 
\frac{1}{\sigma^2} \left\| \mathbf{i}[\mathbf{p} - \left( \mathbf{t} + \mathbf{U}\mathbf{c} \right) \right\|
\end{align}

Equation \ref{} can also be solved using the previously described algorithm by properly accounting for the new regularization terms.

The solution for \emph{inverse} algorithms, i.e.: SIC, Fast-SIC, AIC and POIC involves slightly different updates for $\delta \mathbf{p}$ due to the ... As noted by the authors of \cite{}, these algorithms require the computation of the parameters jacobian matrix $\mathbf{J}_{\mathbf{p}}$ which converts the original inverse compositional incremental updates to its first-order forward equivalent. The update is given by

\begin{equation}
\delta \mathbf{p} = 
\left( \sigma^2 \boldsymbol{\Lambda}^{-1} + \mathbf{J}_{\mathbf{p}} \left( \mathbf{J}^T\mathbf{J} \right)^{-1} \mathbf{J}_{\mathbf{p}}^T \right)^{-1} 
\left( \sigma^2 \boldsymbol{\Lambda}^{-1} \mathbf{p} + \mathbf{J}_{\mathbf{p}} \delta \mathbf{p}^{-1} \right)
\end{equation}

where

\begin{equation}
\delta \mathbf{p}^{-1} = 
\left( \mathbf{J}^T\mathbf{J} \right)^{-1} \mathbf{J}^T \left( \mathbf{i}[\mathbf{p}] - \mathbf{t}_{\mathbf{c}_o} \right)
\end{equation}

\subsubsection{Bayesian Inference}

Finally, a Bayesian Inference procedure for fitting AAMs based on the same probabilistic AAM used here was proposed by the authors of \cite{}. The key idea consist of deriving and expression for the marginalized appearance density $p(\mathbf{i}[\mathbf{p}] | \mathbf{p}, \Theta)$ by integrating out the latent appearance subspace, i.e. the appearance parameters

\begin{align}
p(\mathbf{i}[\mathbf{p}] | \mathbf{p}, \Theta) & = \int p(\mathbf{i}[\mathbf{p}] | \mathbf{p}, \mathbf{c}, \Theta) p(\mathbf{c}) d \mathbf{c}
\\
& = \mathcal{N} \left( \mathbf{t}, \mathbf{U}\boldsymbol{\Sigma}\mathbf{U}^T + \sigma^2 \mathbf{I} \right)
\end{align}

\begin{align}
\mathbf{p}_o & = \underset{\mathbf{p}}{\mathrm{arg\,max\;}} 
p(\mathbf{p}, \mathbf{i}[\mathbf{p}] | \mathbf{p}, \Theta)
\\
& = \underset{\mathbf{p}}{\mathrm{arg\,max\;}} 
p(\mathbf{p}) p(\mathbf{i}[\mathbf{p}] | \mathbf{p}, \Theta)
\\
& = \underset{\mathbf{p}}{\mathrm{arg\,max\;}} 
p(\mathbf{p}) \int p(\mathbf{i}[\mathbf{p}] | \mathbf{p}, \mathbf{c}, \Theta) p(\mathbf{c}) d \mathbf{c}
\\
& = \underset{\mathbf{p}}{\mathrm{arg\,max\;}} 
\ln p(\mathbf{p}) + \ln \left( \int p(\mathbf{i}[\mathbf{p}] | \mathbf{p}, \mathbf{c}, \Theta) p(\mathbf{c}) d \mathbf{c} \right)
\\
& = \underset{\mathbf{p}}{\mathrm{arg\,min\;}} 
-\ln p(\mathbf{p}) - \ln \left( \int p(\mathbf{i}[\mathbf{p}] | \mathbf{p}, \mathbf{c}, \Theta) p(\mathbf{c}) d \mathbf{c} \right)
\\
& = \underset{\mathbf{p}}{\mathrm{arg\,min\;}} 
\left\| \mathbf{p} \right\|_{\boldsymbol{\Lambda}^{-1}} +
\left\| \mathbf{i}[\mathbf{p}] - \mathbf{t} \right\|_{\left( \mathbf{U}\boldsymbol{\Sigma}\mathbf{U}^T + \sigma^2 \mathbf{I} \right)^{-1}}
\end{align}


