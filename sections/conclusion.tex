\section{Conclusion}
\label{sec:conclusion}

In this paper we thoroughly studied the problem of fitting AAMs using CGD algorithms. We presented a unified and complete view of these algorithms and classified them with respect to three of their main characteristics:
\begin{itemize} 
	\item the \emph{cost function} defining the fitting problem; 
	\item the type of \emph{composition} used; and 
	\item the \emph{optimization method} employed to solve the non-linear optimization problem. 
\end{itemize}

Furthermore, we extended the previous view by:
\begin{itemize}
	\item Proposing a novel \emph{Bayesian cost function} for fitting AAMs that can be interpreted as a more general formulation of the well-known project-out loss. We assumed a probabilistic model for appearance generation with both Gaussian noise and a Gaussian prior over a latent appearance space. Marginalizing out the latent appearance space, we derived a novel cost function that only depends on shape parameters and that can be interpreted as a valid and more general probabilistic formulation of the well-known project-out cost function \cite{Matthews2004}. In the experiments, we showed that our Bayesian formulation considerably outperforms the original project-out cost function.
	
	\item Proposing \emph{asymmetric} and \emph{bidirectional} compositions for CGD algorithms. We showed the connection between Gauss-Newton Asymmetric algorithms and Efficient Second-order Minimization (ESM) algorithms and experimentally proved that these two novel types of composition lead to better convergent and more robust CGD algorithm for fitting AAMs.
	
	\item Providing new valuable insights into existent CGD algorithms by reinterpreting them as direct applications of the \emph{Schur complement} and the \emph{Wiberg method}.
\end{itemize} 

Finally, we made the implementation of all algorithms studied in this paper publicly available as part of the Menpo Project\footnote{\url{http://www.menpo.org}}.

In terms of future work, we plan to:
\begin{itemize}
	\item Adapt existent Supervised Descent (SD) algorithms for face alignment \cite{Xiong2013, Tzimiropoulos2015} to AAMs and investigate their relationship with the CGD algorithms studied in this paper. 
	
	\item Investigate if our Bayesian cost function and the proposed asymmetric and bidirectional compositions can also be successfully applied to similar generative parametric models, such as the Gauss-Newton Parts-Based Deformable Model (GN-DPM) proposed in \cite{Tzimiropoulos2014}.   
\end{itemize} 