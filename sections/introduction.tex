\section{Introduction}
\label{sec:intro}

Active Appearance Models (AAMs) \cite{Cootes2001, Matthews2004} are one of the most popular and well-established techniques for modeling and segmenting deformable objects in computer vision. AAMs are generative parametric models of shape and appearance that can be \emph{fitted} to images to recover the set of model parameters that best describe a particular instance of the object being modeled.

Fitting AAMs is a non-linear optimization problem that requires the  minimization (maximization) of a global error (similarity) measure between the input image and an initial model instance. Several approaches \cite{Cootes2001, Hou2001, Matthews2004, Batur2005, Gross2005, Donner2006, Papandreou2008, Liu2009, Saragih2009, Amberg2009, Tresadern2010, Martins2010, Sauer2011, Tzimiropoulos2013, Kossaifi2014, Antonakos2014} have been proposed to define and solve the previous optimization problem. Broadly speaking, the they can be divided into two different groups: 
\begin{itemize}
\item \emph{Regression} based \cite{Cootes2001, Hou2001, Batur2005, Donner2006, Saragih2009, Tresadern2010, Sauer2011}
\item \emph{Optimization} based \cite{Matthews2004, Gross2005, Papandreou2008, Amberg2009, Martins2010, Tzimiropoulos2013, Kossaifi2014}
\end{itemize}

Regression based techniques attempt to solve the problem by learning a direct function mapping between the error measure and the optimal values of the parameters. Most notable approaches include variations on the original \cite{Cootes2001} fixed linear regression approach of \cite{Hou2001, Donner2006}, the adaptive linear regression approach of \cite{Batur2005}, and the works of \cite{Saragih2009} and \cite{Tresadern2010} which considerably improved upon previous techniques by using boosted regression. Also, Cootes and Taylor \cite{Cootes2001b} and Tresadern et al. \cite{Tresadern2010} showed that the use of non-linear gradient-based and Haar-like appearance representations, respectively, lead to better fitting accuracy in regression based AAMs. 

Optimization based methods for fitting AAMs were proposed by Matthews and Baker in \cite{Matthews2004}. These techniques are known as Compositional Gradient Decent (CGD) algorithms and are based on direct analytical optimization of the error measure. Popular CGD algorithms include the very fast project-out Inverse Compositional (PIC) algorithm \cite{Matthews2004}, the accurate but slow Simultaneous Inverse Compositional (SIC) algorithm \cite{Gross2005}, and the more efficient versions of SIC presented in \cite{Papandreou2008} and \cite{Tzimiropoulos2013}. On the other hand, Lucey et al. \cite{Lucey2013} extended these algorithms to the Fourier domain increasing their robustness; and the authors of \cite{Antonakos2014} showed that optimization based AAMs using non-linear feature based (e.g. SIFT\cite{Lowe1999} and HOG \cite{Dalal2005}) appearance models were competitive with modern state-of-the-art techniques in non-rigid face alignment \cite{Xiong2013, Asthana2013} in terms of fitting accuracy.

AAMs have been often criticized due to several reasons: 
\begin{inparaenum}[\itshape i\upshape)] 
\item the limited representational power of their linear appearance model; 
\item the difficulty of optimizing shape and appearance parameters simultaneously; and
\item the complexity involved in handling occlusions. 
\end{inparaenum}
However, recent works in this area \cite{Papandreou2008, Saragih2009,Tresadern2010, Lucey2013, Tzimiropoulos2013, Antonakos2014} suggest that these limitations might have been over-stressed in the literature and that AAMs can produce highly accurate results if appropriate training data \cite{Tzimiropoulos2013}, appearance representations \cite{Tresadern2010, Lucey2013, Antonakos2014} and fitting strategies \cite{Papandreou2008, Saragih2009, Tresadern2010, Tzimiropoulos2013} are employed.

In this paper, we study the problem of fitting AAMs using CGD algorithms thoroughly. Summarizing, our main contributions are:
\begin{itemize}
	\item To present a unified and complete overview of the most relevant and recently published CGD algorithms for fitting AAMs \cite{Matthews2004, Gross2005, Papandreou2008, Amberg2009, Martins2010, Tzimiropoulos2012, Tzimiropoulos2013, Kossaifi2014}. To this end, we classify CGD algorithms with respect to three main characteristics: 
	\begin{inparaenum}[\itshape i\upshape)] 
		\item the \emph{cost function} defining the fitting problem; 
		\item the type of \emph{composition} used; and 
		\item the \emph{optimization method} employed to solve the non-linear optimization problem. 
	\end{inparaenum}

	\item To review the probabilistic interpretation of AAMs and propose a novel \emph{Bayesian formulation}\footnote{A preliminary version of this work \cite{Alabort2014} was presented at CVPR 2014.} of the fitting problem. We assume a probabilistic model for appearance generation with both Gaussian noise and a Gaussian prior over a latent appearance space. Marginalizing out the latent appearance space, we derive a novel cost function that only depends on shape parameters and that can be interpreted as a valid and more general probabilistic formulation of the well-known project-out cost function \cite{Matthews2004}. Our Bayesian formulation is motivated by seminal works on probabilistic component analysis and object tracking \cite{Moghaddam1997, Roweis1998, Tipping1999}.

	\item To propose the use of two novel types of composition for AAMs:
	\begin{inparaenum}[\itshape i\upshape)] 
		\item \emph{asymmetric}; and 
		\item \emph{bidirectional}. 
	\end{inparaenum} These types of composition have been widely used in the related field of parametric image alignment \cite{Malis2004, Megret2008, Autheserre2009, Megret2010} and use the gradients of both image and appearance model to derive better convergent and more robust CGD algorithms.

	\item To provide valuable insights into existent add-hoc strategies used to derived fast and exact simultaneous algorithms for fitting AAMs by reinterpreting them as direct applications of the \emph{Schur complement} \cite{Boyd2004} and the \emph{Wiberg method} \cite{Okatani2006, Strelow2012}.
\end{itemize}

The reminder of the paper is structured as follows. Section \ref{sec:aam} introduces AAMs and reviews their probabilistic interpretation. Section \ref{sec:fitting} constitutes the main section of the paper and contains the discussion and derivations related to the cost functions \ref{sec:cost_function}; composition types \ref{sec:composition}; and optimization methods \ref{sec:optimization}. Implementation details are provided in Section \ref{sec:implementation} and experimental results reported in Section \ref{sec:experiment}. Finally, conclusions are drawn in Section \ref{sec:conclusion}.

